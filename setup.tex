
\section{Protocol Setup}\label{sec:hpsetup}
In order to create a head-protocol instance, an initiator invites a set of
participants (himself being one of them) to join by announcing to them the
protocol parameters: the list of participants, the parameters of the
(multi-)signature scheme to be used, etc.

\begin{itemize}
  \item Each party establishes pairwise communication channels to all other parties. That is, every network message received from a specific party is checked for (channel) authentication. It is the implementer’s duty to find a suitable authentication process for the communication channels.

  \item For some digital-signature scheme, each party $\party_i$ generates a
        (``Cardano'') key pair $(\sVKI{i},\sSKI{i})$ and sends his respective
        verification key $\sVKI{i}$ to all other parties. This standard
        digital-signature scheme will be used to authenticate mainchain
        transactions that are restricted to members of the head-protocol
        instance. In the case of Cardano, these are Ed25519 keys.

  \item For the multisignature scheme (MS) \todo{see Section~\ref{sec:multisig}}
        each party $\party_i$ generates a (``Hydra'') key pair
        \[
          (\daPuI{i},\daPrI{i})  \gets \ \msKeyGen(\msParams)
        \]
        and sends their hydra verification key $\daPuI{i}$ to all other parties. This
        multi-signature scheme may or may not be compatible with the standard
        signature scheme above. The multisignature scheme will be used for the
        offchain confirmation (and offchain and onchain verification) of
        head-protocol events.

  \item Each party then computes the aggregate key from the received
        verification keys:
        \[
          \hpAK \ \gets \ \msCombVK(\msParams,(\daPuI{j})_{j\in[n]}) \,.
        \]
        Note that this aggregation may be a simple concatenation.

  \item Each party $\party_i$ stores their signing key and all received
        verification keys for both the Cardano (standard) and Hydra
        (multi-)signature schemes.
\end{itemize}

If any of the above fails (or the party does not agree to join the head in the
first place), the party aborts the initiation protocol and ignores any further
action.\footnote{Of course, aborting the initiation can be achieved more
  gracefully by explicitly notifying the initiator about one's
  non-participation. Techniques are even known to finish such an initiation in
  agreement among all parties~\cite{PODC:FGHHS02}.} Finally, one of the
participants, who will become the initiator, posts the \mtxInit{} transaction
onchain as described next in Section~\ref{sec:mainchain}.


% TODO: More compact?
% The head protocol uses a multisignature (MS) scheme
% $\ms = (\msSetup,\allowbreak\msKeyGen,\allowbreak\msCombVK,\allowbreak
% \msSign, \allowbreak\msComb, \allowbreak\msVfy)$ (see
% Section~\ref{sec:prel}).
% The setup algorithm $\hpSetup$ simply
% generates the public parameters for the MS scheme, i.e., it runs
% $\msParams \gets \msSetup(1^\spara)$ and outputs
% $\hpParams := \msParams$.  The key-generation algorithm
% $\hpKG(\hpParams)$ (run by every party independently) uses
% $\hpParams = \msParams$ to generate a key pair, i.e., it outputs
% \[
%   (\daPu,\daPr) %
%   \ := \ (\msVK,\msSK) %
%   \ \gets \ \msKeyGen(\msParams) \ .
% \]
% The aggregation algorithm $\hpAgg$ uses the aggregation mechanism
% $\msCombVK$ of the multisignature scheme to combine public keys
% $\{\hpPui i\}_i = \{\msVK_i\}_i$ into an aggregated public key
% \[
%   \hpAK \ := \ \msCVK \ \gets \ \msCombVK(\{\msVK_i\}_i) \ .
% \]

%%% Local Variables:
%%% mode: latex
%%% TeX-master: "main"
%%% End:
