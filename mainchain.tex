\section{Mainchain}
\label{sec:mainchain}

Here we describe the details of the mainchain state
machine (SM) controlling a Hydra head (see
Fig.~\ref{fig:SM_states_basic}).  For state transitions, a formal
description of the conditions in $\cemTxCon$ is foregone in favor of
the intuitive explanations in the text and the figures.

\dparagraph{Onchain verification algorithms.} The status of the head
is maintained in a variable $\eta$, which is part of the SM state and
updated by so-called \emph{onchain verification (OCV) algorithms}
$\ocvInitial$, $\ocvClose$, $\ocvContest$, and $\ocvFinal$. In the context
of the mainchain protocol, these OCV algorithms are intentionally kept
as generic as possible; this keeps the mainchain SM compatible with
many potential head-protocol variants.
The concrete OCV algorithms for the head protocol specified in this
paper are given in context of the head protocol itself as they depend
on the specific head-protocol internals: verification of head-protocol
certificates and related onchain state updates.
As such, the OCV algorithms can be seen as abstract mainchain algorithms
implemented by the specific head protocol. Consequently, the OCV
implementation for our head protocol is described in Section~\ref{sec:hpocv}.

\begin{figure}[h]

  \centering

  % \includegraphics[width=\textwidth/2,trim=130 330 430 50,clip]{figures/SM_commit_tx.pdf}

  % TODO: clean draw marked up version
  \includegraphics[width=\textwidth*2/3]{figures/SM_commit_tx.png}

  \caption{
    \mtxInit{} transaction (left) with one \mtxCom{} transaction
    (right) attached locking one output (center).}\label{fig:SM_commit_tx}

\end{figure}


%%% Local Variables:
%%% mode: latex
%%% TeX-master: "main"
%%% End:


\dparagraph{Initial state.}  After the setup phase of Section~\ref{sec:hpsetup},
the head initiator posts an \mtxInit{} transaction (see
Fig.~\ref{fig:SM_commit_tx}).  The \mtxInit{} transaction
establishes the SM's initial state
\(
  (\stInitial,\hpAK,\hMT,\nop,\cPer),
\)
where $\stInitial$ is a state identifier, $\hpAK$ is the aggregated multisignature
key established during the setup phase, $\hMT$ is the root of a Merkle tree for
the signature verification keys $\hppuv=(k_1,\ldots,k_\nop)$ exchanged during the
setup phase (identifying the head members), $\nop$
is the number of head members, and $\cPer$ is the length of the
contestation period.  The \mtxInit{} transaction also forges $\nop$
participation tokens $\{p_1,\ldots,p_\nop\} :: \mathsf{cid}$, where
the currency ID $\mathsf{cid}$ is given by the unique \emph{monetary-policy script}
consumed by the $\mathsf{cid}$ input. The script is unique as it is bound to
an output and the ledger prevents double spending. Consequently, we can
use $\mathsf{cid}$ as a unique identifier for the newly initialized head.

Crucially, the \mtxInit{} transaction has $\nop$ outputs, where each
output is locked by a validator $\nuSetup$ and the $\ith i$ output has
$k_i$ in its data field.  Validator $\nuSetup$ ensures the following:
either the output is consumed by \begin{menumerate} \item  an SM abort transaction (see
below) or \item a commit transaction (identified by having validator
$\nuCom$ in its only output), and
\begin{menumerate}
\item the transaction is signed and the signature verifies as valid with verification key $k_i$,
\item the data field of the output of the commit transaction is
  $U_i = \makeUTxO(o_1,\ldots,o_m)$, where the $o_j$ are the outputs
  referenced by the commit transaction's inputs and $\makeUTxO$
  stores pairs $(\txOutRef_j,o_j)$ of outputs $o_j$ with the
  corresponding output reference $\txOutRef_j$.
\end{menumerate}
\end{menumerate}

% Note that all the parameters of the head and all keys are agreed upon
%out-of-band (see also the head protocol). %
%\footnote{Note that the head protocol includes algorithms for
%  generating the keys $k_i$ and an aggregated key $\hpAK$.} %
The general well-formedness and validity of the \mtxInit{}
transaction is checked on the mainchain. The head members additionally check whether the head parameters
match the parameters agreed on during the setup phase. In case of a
mismatch the head opening is considered as failed.


\dparagraph{Committing outputs to a head.}
To lock outputs % designated
for a Hydra head, the $\ith i$
head member will attach a \mtxCom{} transaction (see
Fig.~\ref{fig:SM_commit_tx}) to the $\ith i$ output of the
\mtxInit{} transaction.  Validator $\nuCom$ ensures that the \mtxCom{}
transaction correctly records the partial UTxO set $U_i$ committed
by the party.

All \mtxCom{} transactions will in turn be collected by an SM
transaction---either \mtxCCom{} or \mtxAbort{} (see below).


\dparagraph{Collecting commits.} The SM transition from $\stInitial$
to $\stOpen$ is achieved by posting the \mtxCCom{} transaction (see
Fig.~\ref{fig:SM_initial_open}).  All parameters $\hpAK$, $\hMT$,
$\nop$, and $\cPer$ remain part of the state, but in addition, a value
$\eta \gets \ocvInitial(U_1,\ldots,U_n)$ is stored in the state.
The idea is that $\eta$ stores information about the initial UTxO set,
which is made up of the individual UTxO sets $U_i$ collected from the
commit transactions, in order to verify head-status information later
(see below).

It is also required that all $\nop$ participation tokens be present in
the SM output of the \mtxCCom{} transaction.  This ensures that the
\mtxCCom{} transaction collects all $\nop$ commit transactions.  Note
that since $\nuSetup$ does not allow an SM commit transaction to
consume the outputs of the initial transaction, the only way to post
the \mtxCCom{} transaction is if each head member has posted a commit
transaction.


\begin{figure}[t!]

  \centering

  %\includegraphics[width=\textwidth/2-2em,trim=420 80 120 300,
  %clip]{figures/SM_initial_open.pdf}

  \includegraphics[width=\textwidth/2]{figures/SM_initial_open.pdf}

  \caption{\mtxInit{} transaction (left) with \mtxCCom{} transaction
    (right) and \mtxCom{} transactions (center).}
  \label{fig:SM_initial_open}

\end{figure}



%%% Local Variables:
%%% mode: latex
%%% TeX-master: "main"
%%% End:



Finally, note that the transition requires a proof $\piMT$ that the
signer $k'$ is in the Merkle Tree belonging to $\hMT$, which ensures
that only head members can post SM transactions.  This will be the
case for all transitions considered in this paper (and will not be
pointed out any further).


\dparagraph{Aborting a head.} The \mtxAbort{} transaction
(see Fig.~\ref{fig:SM_initial_final}) allows a party to abort the
creation of a head in case some parties fail to post a commit
transaction.  The final state does not contain any information (beyond
its identifier), but it is ensured that (1) the outputs $U$ correspond
to the union of all committed UTxO sets $U_i$ and (2) all
participation tokens are burned.


\begin{figure}

  \centering

  % \includegraphics[width=\textwidth/2-2em,trim=350 20 240 300,
  % clip]{figures/SM_initial_final.pdf}
  % TODO: clean draw marked up version
  \includegraphics[width=\textwidth/2]{figures/SM_initial_final.png}

  \caption{\mtxInit{} transaction (left) with \mtxAbort{} transaction (right)
    and \mtxCom{} transactions (center).}\label{fig:SM_initial_final}

\end{figure}



%%% Local Variables:
%%% mode: latex
%%% TeX-master: "main"
%%% End:



\dparagraph{Close transaction.} In order to close a head, a head
member may post the \mtxClose{} transaction (see
Fig.~\ref{fig:SM_open_closed}), which results in a state transition
from the $\stOpen$ state to the $\stClosed$ state.  For a successful
close, a head member must provide valid information $\xi$ about (their
view of) the current head state.  This information is passed through
OCV algorithm $\ocvClose$, resulting in a new OCV status
$\eta' \gets \ocvClose(\hpAK,\eta,\xi)$.  OCV algorithm $\ocvClose$
uses the previous OCV status $\eta$ and $\hpAK$ to check the head
information~$\xi$.  Note that if a check fails, $\ocvClose$ may output
$\bot$, but in order for a \mtxClose{} transaction to be valid,
$\eta' \neq \bot$ is required.

Once a \mtxClose{} transaction has been posted, a \emph{contestation period}
begins which should last at least $\cPer$ slots.  Hence, the last slot
$\Tfinal$ of the contestation period is recorded in the state, and it
is ensured that $\Tfinal \geq \txRmax' + \cPer$.

Finally, the SM state is extended by a set $\contesters$
initialized to the poster's signing key, i.e.,
$\contesters \gets \{k'\}$.   $\contesters$ is used to ensure that no party
posts more than once during the contestation period.


\begin{figure}

  \centering

  % \includegraphics[width=\textwidth/2]{figures/SM_open_closed.pdf}
  % TODO: clean draw marked up version
  \includegraphics[width=\textwidth/2]{figures/SM_open_closed.png}

  \caption{\mtxCCom{} transaction (left) with \mtxClose{} transaction
    (right).}\label{fig:SM_open_closed}

\end{figure}



%%% Local Variables:
%%% mode: latex
%%% TeX-master: "main"
%%% End:



\dparagraph{Contestation.} If the party first closing a head posts
outdated/incomplete information about the current state of the head,
any other party may post a \mtxContest{} transaction (see
Fig.~\ref{fig:SM_closed_closed}), which causes a state transition from
the $\stClosed$ state to itself.  The transition handles update
information $\xi$ by passing it through OCV algorithm $\ocvContest$,
resulting in a new OCV status
$\eta' \gets \ocvContest (\hpAK,\eta,\xi)$.  OCV algorithm
$\ocvContest$ uses the previous OCV status $\eta$ and $\hpAK$ to check
the update information $\xi$.  Similarly to $\ocvClose$, $\ocvContest$
may output $\bot$, but in order for a \mtxContest{} transaction to be
valid $\eta' \neq \bot$ is required.

The \mtxContest{} transaction is only valid if the old set
$\contesters$ of parties who have contested (or closed) so far does not yet
include the poster, i.e., $k' \notin \contesters$.  If this check
passes, the set is extended to include the poster of the \mtxContest{}
transaction, i.e., $\contesters' \gets \contesters \cup \{k'\}$.
Furthermore, \mtxContest{} transactions may only be posted up until
$\Tfinal$, i.e., it is required that $\txRmax' \leq \Tfinal$.

Observe that during the contestation period, up to $\nop-1$
\mtxContest{} transactions may be posted (of course, the parameter
$\cPer$ has to be chosen large enough as to allow each head member to
potentially post a \mtxClose{}/\mtxContest{} transaction).


\begin{figure}

  \centering

  %\includegraphics[width=\textwidth/2-2em,trim=280 120 160 260,
  %clip]{figures/SM_closed_closed.pdf}

  \includegraphics[width=\textwidth/2]{figures/SM_closed_closed.pdf}

  \caption{\mtxClose{}/\mtxContest{} transaction (left);
    \mtxContest{} transaction (right)}
  \label{fig:SM_closed_closed}

\end{figure}



%%% Local Variables:
%%% mode: latex
%%% TeX-master: "main"
%%% End:


\begin{figure}[t!]

  \centering

  %\includegraphics[width=\textwidth/2-2em,trim=250 40 240 280,
  %clip]{figures/SM_closed_final.pdf}

  \includegraphics[width=\textwidth/2]{figures/SM_closed_final.pdf}

  \caption{\mtxClose{}/\mtxContest{} transaction (left);
    \mtxFanout{} transaction (right)}
  \label{fig:SM_closed_final}

\end{figure}



%%% Local Variables:
%%% mode: latex
%%% TeX-master: "main"
%%% End:


\dparagraph{Final state.}  Once the contestation phase is over, a head
may be finalized by posting a \mtxFanout{} transaction, taking the SM
from $\stClosed$ to $\stFinal$.  The \mtxFanout{} transaction must
have outputs that correspond to the most recent head state.  To that
end, OCV predicate $\ocvFinal$ checks the transaction's output set $U$
against the information recorded in $\eta$.  The \mtxFanout{}
transaction is only valid if $\ocvFinal$ outputs $\true$.  Moreover,
to ensure that the \mtxFanout{} transaction is not posted too early,
$\txRmin' > \Tfinal$ is required.  Finally, all participation tokens
must be burned.


%%% Local Variables:
%%% mode: latex
%%% TeX-master: "main"
%%% End:
