\section{Off-Chain Protocol}\label{sec:offchain}

This section describes the actual Coordinated Hydra Head protocol, an even more
simplified version of the original publication~\cite{hydrahead20}. See the
protocol overview in Section~\ref{sec:overview} for an introduction and notable
changes to the original protocol. While the on-chain part already describes the
full life-cycle of a Hydra head on-chain, this section completes the picture by
defining how the protocol behaves off-chain and notably how certain values
relate between on- and off-chain semantics. The protocol is specified as a
reactive system that processes three kinds of events:
\begin{enumerate}
  \item on-chain protocol transactions as introduced in the previous section (\ref{sec:on-chain})
  \item off-chain network messages sent between protocol actors (parties):
    \begin{itemize}
      \item $\hpRT$: to request a transaction to be included in the next snapshot
      \item $\hpRS$: to request a snapshot to be created \& signed by every head member
      \item $\hpAS$: to acknowledge a snpashot by replying with their signatures
    \end{itemize}
  \item client commands as received from the environment
    \begin{itemize}
      \item $\mathtt{init}$: to start initialization of a head
      \item $\mathtt{new}$: to ingest a new transaction to an open head
      \item $\mathtt{close}$: to request closure of an open head
    \end{itemize}
\end{enumerate}

\todo{by extending the figure it is describing more states now -> better state diagram?}
The behavior is fully specified with corresponding sections in
Figure~\ref{fig:head_coordinated} using variables and notation introduced in the
following paragraphs.

\subsection{Assumptions}
\todo{move to protocol setup?}
\begin{itemize}
  \item Every network message received from a specific party is checked for
        authentication. An implementation of the specification needs to find a
        suitable means of (channel) authentication. Unauthenticated messages are
        dropped and not yield to an event.
  \item The head protocol gets correctly (and with completeness) notified about
        observed transactions on-chain belonging to the respective head
        instance.
  \item The specification covers only a single instance of a Hydra head.
        However, some implementations may choose to track multiple instances. As
        multiple Hydra heads might exist on the same blockchain, it is vital
        that they do not interfere and the specification will take special care
        to ensure this.
  \item All events are processed to completion, i.e.\ run-to-completion semantics
        without preemption.
  \item Events are deduplicated. That is, any two identical events must not lead
        to multiple invocations of the handling semantics.
  \item Given the specification, events may pile up forever and implementations
        need to consider these situations (i.e.\ potential for DoS). Note that,
        from a security standpoint, these situations are identical to a
        non-collaborative peer.
  \item The lifecycle of a Hydra head on-chain does not cross (hard fork)
        protocol update boundaries. Note that these events are announced in
        advance hence it should be possible for implementations to react in such
        a way as to expedite closing of the head before such a protocol update.
        This further assumes that the contestation period parameter is picked
        accordingly.
\end{itemize}

\subsection{Notation}
\begin{itemize}
  \item $\KwOn~(event,\ldots)$ specifies how the protocol reacts on a given event. Further information may be available via arguments and origin of the event.
  \item $\Req~P$ means that logic expression $P$ must be satisfied for the further execution of
        a routine, while discontinued on $\neg P$.
  \item $\KwWait~P$ is a \todo{blocking in GDoc?} non-blocking wait for
predicate $P$ to be satisfied. On $\neg P$, the execution of the
routine is stopped, queued, and reactivated at latest when $P$ is
satisfied.
  \item $\Fail$ implies detection of cheating (i.e., head can be closed), no
        further requests or acknowledgments must be sent.
  \item $\Multi{}$ means that a message is (channel-) authenticated and sent to all participants of this head, including the sender.
  \item $\PostTx{}$ has a party create and submit the given transaction on-chain. See Section~\ref{sec:on-chain} for individual transaction details.
\end{itemize}



\subsection{Variables}

\begin{center}
\begin{tabular}{|l|l|}\hline
  $\hats$  & Sequence number of seen snapshot. \\ \hline
  $\bars$  & Sequence number of confirmed snapshot. \\ \hline
  $\hatmU$ & Seen snapshot object. \\ \hline
  $\barmU$ & Confirmed snapshot object. \\ \hline
  $\hatmL$ & Local ledger extension using seen transactions to prepare the next snapshot.\\ \hline
  $\mT$    & Set of all transactions ever received via reqTx (independent of their validity or mutual conflicts).\\  \hline
  $\hatmT$ & Set of transactions that extend the $\barmU$ (or $\hatmU$) to form $\hatmL$. \\
           & Transaction candidates that go into the next snapshot (if this party is the next leader).\\ \hline
  $\barmT$ & Set of all confirmed transactions up to the latest confirmed snapshot.\\  \hline
\end{tabular}
\end{center}

The following figure visualizes the possible event flows from the view of a single party.
\begin{figure}[h]
  \centering
  \includegraphics[width=\linewidth*2/3]%
                  {fig/reqackstates.png}
                  \caption{State diagram of reqSn/ackSn cycles.}
                  \label{fig:sim:baselines}
\end{figure}

\begin{figure*}[t!]

  \def\sfact{0.8}
  \centering
  \begin{algobox}{Coordinated Hydra Head}
    \medskip
    \begin{tabular}{c}
      \begin{tabular}{cc}
        \adjustbox{valign=t,scale=\sfact}{
         \begin{walgo}{0.65}
          %%% INIT
          \On{$(\hpInit,i,\hpPuv,\hpPr,\Uinit)$ %
            from client}{ %
            % let $i$ be position of $\hpPr$'s public key in $\hpPuv$
            % \; %
            $\msVKL \gets \hpPuv$ \; %
            $\msCVK \gets \msCombVK(\msVKL)$ \; %
            $\msSK \gets \hpPr$ \; %c

            $\hats,\bars \gets 0$ \; %
            $\hatmU, \barmU \gets \Sno (0,\Uinit)$ \; %
            % $\hatmL,\barmL \gets \Uinit$ \; %
            $\hatmL \gets \Uinit$ \; %
            % $\hatmT,\barmT \gets \emptyset$ \; %
            $\mT, \hatmT,  \barmT \gets \emptyset$ \; % all, seenforsnap, confirmedforsnap
          }

          
          \vspace{12pt} %
          
          %%% NEW TX
          \On{$(\hpNew,\tx)$ %
            from client}{ %
            \Req $\validTx(\tx)$ and %
            $\hatmL \applytx \tx \neq \bot$ \; %

            % $\hash \gets \hashF(\tx)$ \; %

            % $\hySignedT[\hash] \gets \hyTxo(i,\tx)$ \; %
            
            \Multi $(\hpRT,\tx)$ \; %
          }           

        \end{walgo}
        }
        &

        \adjustbox{valign=t,scale=\sfact}{
        \begin{walgo}{0.6}
          %%% NEW SN  
          \On{$(\hpNS)$ %
            from client}{ // assumed to be called only once%%
              
            \Req $\hpLdr(\bars + 1) = i$ %
            and $\hats = \bars$ \;
             
            % $\hySignedU \gets %
            % \hySto(\hyConfSN + 1,\hyConfL,\hyConfT)$ \; %

            %$\hatmT \gets 
            
            \Multi $(\hpRS,\bars+1,\hatmT\hpProjH)$ \; %
          }

          \vspace{12pt} %

          %%% CLOSE
          \On{$(\hpClose)$ from client}{ %
            \Return $(\barmU.U, \barmU.s, \barmU.\msCSig)$ \; %
          }
          
          \vspace{12pt} %
          
          %%% CONTEST
          \On{$(\hpCont,\eta)$ from client}{ %
            $(U_\eta,s_\eta) \gets \eta$ \; %

            \Req $s_\eta > \bars$ %
             
            \Return $(U_\eta,s_\eta)$ \; %
          }

          \vspace{12pt}

          % \On{$(\hpFinalize,\eta)$ from client}{ %
          %   $(\hash,s,T) \gets \eta$ \; %            
          % }    
          

        \end{walgo}
          }
        % &
        
        % \adjustbox{valign=t,scale=\sfact}{
        % \begin{walgo}{0.3}
        %   \If{bla}{
        %     bla \; %
        %   }      
        % \end{walgo}} %

      \end{tabular}
      
      \vspace{-10pt}
      \\
      % \ifdefined\IsLLNCS %
        \multicolumn{1}{l}{\line(1,0){490}} %
      % \else %
      %   \multicolumn{1}{l}{\line(1,0){465}} %
      % \fi %
      \\

      \begin{tabular}{c@{}c}
        \adjustbox{valign=t,scale=\sfact}{
        \begin{walgo}{0.65}
          %%% REQ TX
          \On{$(\hpRT,\tx)$ %
            from $\party_j$}{ %
            \Req $\validTx(\tx)$ and %
            $\hatmL \applytx \tx \neq \bot$ \; %
            
            $\mT[\hash] \gets \tx$ % all seen txs

            % \If{$(\hats\in\{\bars,\bars+1\}) \wedge (\hpLdr(\hats+1) = i) \wedge (\hatmL\applytx\tx\not=\bot)$}{
              $\hatmT[\hash] \gets \mT[\hash]$ % candidates for next snapshot \; %

              $\hatmL \gets \hatmL\applytx\tx$ %

            \If{
                \Req $\hpLdr(\bars + 1) = i$ %
                and $\hats = \bars$ \;
                \Multi $(\hpRS,\bars+1,\hatmT\hpProjH)$ \; %
                }
            
          }

          \vspace{12pt} %

          %%% REQ SN
          \On{$(\hpRS,s,T)$ %
            from $\party_j$}{ %

            \Req $s = \hats + 1$ %
            and $\hpLdr(s) = j$ \; %

            \Wait{$\bars = \hats$ %
              and $\barmU\applytx T \not= \bot$ % meaning reach is 'complete' for extension
            }{ %
               $\hats \gets \bars + 1$ \; %

               $\forall\tx\in\Reach^{\mT}(T)$:
                \Out $(\hpSeen,\tx)$ \; %
              
               $\hatmU \gets
               \Sno(\hats,\barmU\applytx T )$ \; %

               $\msSig_i \gets %
               \msSign(\hyPr, (\hats \| U_{0}.h) \| \hatmU.h)$ \; %
 
               \Send $(\hpAS,\hats,\msSig_i)$ %
               to $\party_j$ \; %

               % \If{$\hpLdr(\hats+1) = i$}{
                 $\hatmT :\subseteq_{\mbox{max}} \mT$ s.t. $\hatmU\applytx\hatmT\not=\bot$ \; %
                 $\hatmL \gets \hatmU\applytx\hatmT$
               % }
            }
           }
          
        \end{walgo}
        }
        &

        \adjustbox{valign=t,scale=\sfact}{
        \begin{walgo}{0.6}
          %%% ACK SN
          \On{$(\hpAS,s,\msSig_j)$ %
            from $\party_j$}{ %

            \Req $s \in \{\hats,\hats+1\}$ \; % 2-round ack/confirmation
            % and $\hySignedU.\hySN = \hySN$ \; %

            \Wait{$\hats=s$
            }{ %
            
            \Req $\hatmU.\hpSigs[j] = \undefined$ \; %

            $\hatmU.\hpSigs[j] \gets \msSig_j$ \; %

            \If{$\forall k: \hatmU.\hpSigs[k] \neq \undefined$}{ %
              $\msCSig %
              \gets \msComb((s \| U_{0}.h) \| \hatmU.\hash, %
              \msVKL, \hatmU.\hpSigs)$ \; %
              
              \If{$\msVfy(\msCVK, (\hats \| U_{0}.h) \| \hatmU.\hash,\msCSig)$}{ %
              $\bars \gets s$ \; %
              $\hatmU.\msCSig \gets \msCSig$ \; %
              $\barmU \gets \hatmU$ \; %
              $\barmT \gets \barmT \cup \barmU.T$ \; %

              $\forall \tx\in T$: \Out $(\hpConf,\tx)$ \; %
  
            }\Else{
             \fail
            }%
            }
           }
          }  
        \end{walgo}

          }
          
        % &
        
        % \adjustbox{valign=t,scale=\sfact}{
        % \begin{walgo}{0.3}
        %   \If{bla}{
        %     bla \; %
        %   }      
        % \end{walgo}} %

      \end{tabular}


    \end{tabular}
    \bigskip
  \end{algobox}
  
  \caption{Head-protocol machine for the \emph{coordinated head} from the perspective of party
    $\party_i$.}
  \label{fig:head_coordinated}

\end{figure*}



%%% Local Variables:
%%% mode: latex
%%% TeX-master: "main"
%%% End:


%%% Local Variables:
%%% mode: latex
%%% TeX-master: "main"
%%% End:
