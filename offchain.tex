\section{Off-Chain Protocol}\label{sec:offchain}

This section describes the actual Coordinated Hydra Head protocol, an even more
simplified version of the original publication~\cite{hydrahead20}. See the
protocol overview in Section~\ref{sec:overview} for an introduction and notable
changes to the original protocol. While the on-chain part already describes the
full life-cycle of a Hydra head on-chain, this section completes the picture by
defining how the protocol behaves off-chain and notably how certain values
relate between on- and off-chain semantics. The protocol is specified as a
reactive system that processes three kinds of events:
\begin{enumerate}
  \item on-chain protocol transactions as introduced in the previous section (\ref{sec:on-chain})
  \item off-chain network messages sent between protocol actors (parties):
    \begin{itemize}
      \item $\hpRT$: to request a transaction to be included in the next snapshot
      \item $\hpRS$: to request a snapshot to be created \& signed by every head member
      \item $\hpAS$: to acknowledge a snpashot by replying with their signatures
    \end{itemize}
  \item client commands as received from the environment
    \begin{itemize}
      \item $\mathtt{init}$: to start initialization of a head
      \item $\mathtt{new}$: to ingest a new transaction to an open head
      \item $\mathtt{close}$: to request closure of an open head
    \end{itemize}
\end{enumerate}

\todo{by extending the figure it is describing more states now -> better state diagram?}
The behavior is fully specified with corresponding sections in
Figure~\ref{fig:head_coordinated} using variables and notation introduced in the
following paragraphs.

\subsection{Assumptions}
\todo{move to protocol setup?}
\begin{itemize}
  \item Every network message received from a specific party is checked for
        authentication. An implementation of the specification needs to find a
        suitable means of (channel) authentication. Unauthenticated messages are
        dropped and not yield to an event.
  \item The head protocol gets correctly (and with completeness) notified about
        observed transactions on-chain belonging to the respective head
        instance.
  \item The specification covers only a single instance of a Hydra head.
        However, some implementations may choose to track multiple instances. As
        multiple Hydra heads might exist on the same blockchain, it is vital
        that they do not interfere and the specification will take special care
        to ensure this.
  \item All events are processed to completion, i.e.\ run-to-completion semantics
        without preemption.
  \item Events are deduplicated. That is, any two identical events must not lead
        to multiple invocations of the handling semantics.
  \item Given the specification, events may pile up forever and implementations
        need to consider these situations (i.e.\ potential for DoS). Note that,
        from a security standpoint, these situations are identical to a
        non-collaborative peer.
  \item The lifecycle of a Hydra head on-chain does not cross (hard fork)
        protocol update boundaries. Note that these events are announced in
        advance hence it should be possible for implementations to react in such
        a way as to expedite closing of the head before such a protocol update.
        This further assumes that the contestation period parameter is picked
        accordingly.
\end{itemize}

\subsection{Notation}
\todo{map access notation}
\begin{itemize}
  \item $\KwOn~(event,\ldots)$ specifies how the protocol reacts on a given event. Further information may be available via arguments and origin of the event.
  \item $\Req~P$ means that logic expression $P$ must be satisfied for the further execution of
        a routine, while discontinued on $\neg P$.
  \item $\KwWait~P$ is a \todo{blocking in GDoc?} non-blocking wait for
predicate $P$ to be satisfied. On $\neg P$, the execution of the
routine is stopped, queued, and reactivated at latest when $P$ is
satisfied.
  \item $\Multi{}$ means that a message is (channel-) authenticated and sent to all participants of this head, including the sender.
  \item $\PostTx{}$ has a party create and submit the given transaction on-chain. See Section~\ref{sec:on-chain} for individual transaction details.
\end{itemize}


\subsection{Variables}
\todo{todo}

\begin{center}
\begin{tabular}{|l|l|}\hline
  $\hats$  & Sequence number of seen snapshot. \\ \hline
  $\bars$  & Sequence number of confirmed snapshot. \\ \hline
  $\hatmU$ & Seen snapshot object. \\ \hline
  $\barmU$ & Confirmed snapshot object. \\ \hline
  $\hatmL$ & Local ledger extension using seen transactions to prepare the next snapshot.\\ \hline
  $\mT$    & Set of all transactions ever received via reqTx (independent of their validity or mutual conflicts).\\  \hline
  $\hatmT$ & Set of transactions that extend the $\barmU$ (or $\hatmU$) to form $\hatmL$. \\
           & Transaction candidates that go into the next snapshot (if this party is the next leader).\\ \hline
  $\barmT$ & Set of all confirmed transactions up to the latest confirmed snapshot.\\  \hline
\end{tabular}
\end{center}

\subsection{Protocol flow}

\subsubsection{Initializing the head}

\dparagraph{$\hpInit$.}\quad Before a head can be initialized, all
parties need to exchange \& agree on protocol parameters during the protocol
setup phase (see Section~\ref{sec:setup}), so we can assume the public Cardano
keys $\hppuv^{setup}$, Hydra keys $\hpAK^{setup}$, as well as the contestation
period $\cPer^{setup}$ are available. One of the clients then can start head initialization using the $\hpInit$ command, which will result in an $\mtxInit$ transaction being posted.\\

\dparagraph{$\mathtt{initialTx}$.}\quad All parties will receive this $\mtxInit$ transaction and validate announced parameters against the pre-agreed $setup$ parameters.\\

\dparagraph{$\mathtt{commitTx}$.}\quad As each party $p_{j}$ posts a
$\mtxCommit$ transaction, the protocol records observed committed UTxO
$U_{p_{j}}$. With all committed UTxOs known, the $\eta$-state is created using
$\mathsf{Combine}$ and the $\mtxCollect$ transaction is posted. Note that while
each participant might post this transaction, only one of them will be included
in the blockchain as the mainchain ledger prevents double spending. Should any
party want to abort, they would post an $\mtxAbort$ transaction and the protocol
would end at this point.\\

\dparagraph{$\mathtt{collectComTx}$.}\quad Upon observing the $\mtxCollect$
transaction, the parties compute $\Uinit \gets \bigcup_{j=1}^{n} U_{p_j}$ using
previously observed committed UTxOs $U_{p_j}$ and initialize
$\hatmU = \barmU = \hatmL = \Uinit$ with it\todo{check $\eta$ against
  $\Uinit$?}. The initial transaction sets are empty
$\mT = \barmT =\hatmT =\emptyset$, and $\bars = \hats = 0$.

\subsubsection{Processing transactions off-chain}

Transactions are announced and captured in so-called snapshots. Parties generate
snapshots in a strictly sequential round-robin manner. The party responsible for
issuing the $\ith i$ snapshot is the \emph{leader} of the $\ith i$ snapshot.
While the frequency of snapshots in the general Head protocol~\cite{hydrahead20}
was configurable, the Coordinated Head protocol does specify a snapshot to be
created after each transaction.\\

\dparagraph{$\hpNew$.}\quad At any time, by calling $(\hpNew,\tx)$, a head
party can (asynchronously) inject a new transaction $\tx$ to the head protocol.
For this, the transaction must be well-formed ($\validTx$) and applicable to the
current local ledger state: $\hatmL \applytx \tx \neq \bot$. If the checks
pass, a $(\hpRT,\tx)$ request is sent out to all parties.\\

\dparagraph{$\hpRT$.}\quad Upon receiving request $(\hpRT,\tx)$, the transaction
gets recorded in $\mT$ and $\barmT$ using the tx hash $h = H(tx)$ and applied
to the local \emph{seen} ledger state: $\hatmL \applytx \tx$. If there is no
current snapshot ``in flight'' ($\hats = \bars$) and the receiving party $i$ is
the next snapshot leader, a request for the next snapshot $\hpRS$ is sent. Note that only transaction hashes are submitted in this message.\\

\dparagraph{$\hpRS$.}\quad Upon receiving request $(\hpRS,s,T)$ from party
$\party_j$, the receiver $\party_i$ checks that $s$ is the next snapshot number
and that party $\party_j$ is responsible for leading its creation. Party
$\party_i$ then waits until the previous snapshot is confirmed ($\bars = \hats$)
and all transactions referred in $T$ \todo{resolve $T \rightarrow \mT$} are
confirmed. Only then, $\party_i$ increments their seen-snapshot counter $\hats$,
resets the signature accumulator $\Sigma$, and computes the UTxO set $\hatmU$ of
the new (seen) snapshot as $\hatmU \gets \barmU \applytx T$. Then, $\party_i$
creates a signature $\msSig_i$ using their signing key $\hyPr$ on a message
comprised by the initial UTxO set $\Uinit$, the new snapshot UTxO set $\hatmU$,
and the new snapshot's number $\hats$. The signature is sent to \emph{all} head
members via message $(\hpAS,\hats,\msSig_i)$. Note that no UTxO sets have to be
exchanged in this process as the parties all locally compute a new snapshot by
the given transaction hashes. Finally, the pending transaction set $\hatmT$ gets
pruned by the just requested transactions $T$ and the local ledger state
$\hatmL$ is updated accordingly.\\

\dparagraph{$\hpAS$.}\quad Upon receiving acknowledgment $(\hpAS,s,\msSig_j)$,
all participants $\Req$ that it is from an expected snapshot (either the last
seen or + 1), the signature is not yet included in $\Sigma$, and potentially
$\KwWait$ for the corresponding $\hpRS$ such that $\hats = s$. They store the
received signature in the signature accumulator $\Sigma$. If a signature from
each party has been collected, $\party_i$ aggregates the multisignature
$\msCSig$ and $\Req$s it to be valid. \todo{detect cheating (fail) here?} If
everything is fine, the snapshot can be considered confirmed by updating
$\bars=s$ and storing everything for later reference in $\barmU$ \todo{different variable name?}. Similar to the $\hpRT$, if $\party_i$ is the next snapshot leader and there are already transactions to snapshot in $\hatmT$, a corresponding $\hpRS$ is to be distributed.

\subsubsection{Closing the head}

\dparagraph{$\hpClose$.}\quad In order to close a head, a client issues the
$\hpClose$ event which uses the latest confirmed snapshot $\barmU$ to create
\begin{itemize}
  \item the new $\eta$-state from the last confirmed UTxO set (a hash of it) and
        snapshot number, and
  \item the certifiate $\xi$ using the corresponding multi-signature.
\end{itemize}
With $\eta'$ and $\xi$, the $\mtxClose$ transaction can be constructed and
posted. See Section~\ref{sec:close-tx} for details about this transaction.
\todo{check paper for more relevant info (CHI)}\\

\dparagraph{$\mathtt{closeTx}/\mathtt{contestTx}$.}\quad When a party observes
the head getting closed or contested, the $\eta$-state extracted from the
\mtxClose{} or \mtxContest{} transaction represents the latest head status that
has been aggregated onchain so far (by a sequence of \mtxClose{} and
\mtxContest{} transactions). If the last confirmed (off-chain) snapshot is newer
than the observed (on-chain) snapshot number $s_{c}$, an updated $\eta$-state
and certificate $\xi$ is constructed posted in a \mtxContest{} transaction (see
Section~\ref{sec:contest-tx}).

\begin{figure*}[t!]

  \def\sfact{0.8}
  \centering
  \begin{algobox}{Coordinated Hydra Head}
    \medskip
    \begin{tabular}{c}
      \begin{tabular}{cc}
        \adjustbox{valign=t,scale=\sfact}{
         \begin{walgo}{0.65}
          %%% INIT
          \On{$(\hpInit,i,\hpPuv,\hpPr,\Uinit)$ %
            from client}{ %
            % let $i$ be position of $\hpPr$'s public key in $\hpPuv$
            % \; %
            $\msVKL \gets \hpPuv$ \; %
            $\msCVK \gets \msCombVK(\msVKL)$ \; %
            $\msSK \gets \hpPr$ \; %c

            $\hats,\bars \gets 0$ \; %
            $\hatmU, \barmU \gets \Sno (0,\Uinit)$ \; %
            % $\hatmL,\barmL \gets \Uinit$ \; %
            $\hatmL \gets \Uinit$ \; %
            % $\hatmT,\barmT \gets \emptyset$ \; %
            $\mT, \hatmT,  \barmT \gets \emptyset$ \; % all, seenforsnap, confirmedforsnap
          }

          
          \vspace{12pt} %
          
          %%% NEW TX
          \On{$(\hpNew,\tx)$ %
            from client}{ %
            \Req $\validTx(\tx)$ and %
            $\hatmL \applytx \tx \neq \bot$ \; %

            % $\hash \gets \hashF(\tx)$ \; %

            % $\hySignedT[\hash] \gets \hyTxo(i,\tx)$ \; %
            
            \Multi $(\hpRT,\tx)$ \; %
          }           

        \end{walgo}
        }
        &

        \adjustbox{valign=t,scale=\sfact}{
        \begin{walgo}{0.6}
          %%% NEW SN  
          \On{$(\hpNS)$ %
            from client}{ // assumed to be called only once%%
              
            \Req $\hpLdr(\bars + 1) = i$ %
            and $\hats = \bars$ \;
             
            % $\hySignedU \gets %
            % \hySto(\hyConfSN + 1,\hyConfL,\hyConfT)$ \; %

            %$\hatmT \gets 
            
            \Multi $(\hpRS,\bars+1,\hatmT\hpProjH)$ \; %
          }

          \vspace{12pt} %

          %%% CLOSE
          \On{$(\hpClose)$ from client}{ %
            \Return $(\barmU.U, \barmU.s, \barmU.\msCSig)$ \; %
          }
          
          \vspace{12pt} %
          
          %%% CONTEST
          \On{$(\hpCont,\eta)$ from client}{ %
            $(U_\eta,s_\eta) \gets \eta$ \; %

            \Req $s_\eta > \bars$ %
             
            \Return $(U_\eta,s_\eta)$ \; %
          }

          \vspace{12pt}

          % \On{$(\hpFinalize,\eta)$ from client}{ %
          %   $(\hash,s,T) \gets \eta$ \; %            
          % }    
          

        \end{walgo}
          }
        % &
        
        % \adjustbox{valign=t,scale=\sfact}{
        % \begin{walgo}{0.3}
        %   \If{bla}{
        %     bla \; %
        %   }      
        % \end{walgo}} %

      \end{tabular}
      
      \vspace{-10pt}
      \\
      % \ifdefined\IsLLNCS %
        \multicolumn{1}{l}{\line(1,0){490}} %
      % \else %
      %   \multicolumn{1}{l}{\line(1,0){465}} %
      % \fi %
      \\

      \begin{tabular}{c@{}c}
        \adjustbox{valign=t,scale=\sfact}{
        \begin{walgo}{0.65}
          %%% REQ TX
          \On{$(\hpRT,\tx)$ %
            from $\party_j$}{ %
            \Req $\validTx(\tx)$ and %
            $\hatmL \applytx \tx \neq \bot$ \; %
            
            $\mT[\hash] \gets \tx$ % all seen txs

            % \If{$(\hats\in\{\bars,\bars+1\}) \wedge (\hpLdr(\hats+1) = i) \wedge (\hatmL\applytx\tx\not=\bot)$}{
              $\hatmT[\hash] \gets \mT[\hash]$ % candidates for next snapshot \; %

              $\hatmL \gets \hatmL\applytx\tx$ %

            \If{
                \Req $\hpLdr(\bars + 1) = i$ %
                and $\hats = \bars$ \;
                \Multi $(\hpRS,\bars+1,\hatmT\hpProjH)$ \; %
                }
            
          }

          \vspace{12pt} %

          %%% REQ SN
          \On{$(\hpRS,s,T)$ %
            from $\party_j$}{ %

            \Req $s = \hats + 1$ %
            and $\hpLdr(s) = j$ \; %

            \Wait{$\bars = \hats$ %
              and $\barmU\applytx T \not= \bot$ % meaning reach is 'complete' for extension
            }{ %
               $\hats \gets \bars + 1$ \; %

               $\forall\tx\in\Reach^{\mT}(T)$:
                \Out $(\hpSeen,\tx)$ \; %
              
               $\hatmU \gets
               \Sno(\hats,\barmU\applytx T )$ \; %

               $\msSig_i \gets %
               \msSign(\hyPr, (\hats \| U_{0}.h) \| \hatmU.h)$ \; %
 
               \Send $(\hpAS,\hats,\msSig_i)$ %
               to $\party_j$ \; %

               % \If{$\hpLdr(\hats+1) = i$}{
                 $\hatmT :\subseteq_{\mbox{max}} \mT$ s.t. $\hatmU\applytx\hatmT\not=\bot$ \; %
                 $\hatmL \gets \hatmU\applytx\hatmT$
               % }
            }
           }
          
        \end{walgo}
        }
        &

        \adjustbox{valign=t,scale=\sfact}{
        \begin{walgo}{0.6}
          %%% ACK SN
          \On{$(\hpAS,s,\msSig_j)$ %
            from $\party_j$}{ %

            \Req $s \in \{\hats,\hats+1\}$ \; % 2-round ack/confirmation
            % and $\hySignedU.\hySN = \hySN$ \; %

            \Wait{$\hats=s$
            }{ %
            
            \Req $\hatmU.\hpSigs[j] = \undefined$ \; %

            $\hatmU.\hpSigs[j] \gets \msSig_j$ \; %

            \If{$\forall k: \hatmU.\hpSigs[k] \neq \undefined$}{ %
              $\msCSig %
              \gets \msComb((s \| U_{0}.h) \| \hatmU.\hash, %
              \msVKL, \hatmU.\hpSigs)$ \; %
              
              \If{$\msVfy(\msCVK, (\hats \| U_{0}.h) \| \hatmU.\hash,\msCSig)$}{ %
              $\bars \gets s$ \; %
              $\hatmU.\msCSig \gets \msCSig$ \; %
              $\barmU \gets \hatmU$ \; %
              $\barmT \gets \barmT \cup \barmU.T$ \; %

              $\forall \tx\in T$: \Out $(\hpConf,\tx)$ \; %
  
            }\Else{
             \fail
            }%
            }
           }
          }  
        \end{walgo}

          }
          
        % &
        
        % \adjustbox{valign=t,scale=\sfact}{
        % \begin{walgo}{0.3}
        %   \If{bla}{
        %     bla \; %
        %   }      
        % \end{walgo}} %

      \end{tabular}


    \end{tabular}
    \bigskip
  \end{algobox}
  
  \caption{Head-protocol machine for the \emph{coordinated head} from the perspective of party
    $\party_i$.}
  \label{fig:head_coordinated}

\end{figure*}



%%% Local Variables:
%%% mode: latex
%%% TeX-master: "main"
%%% End:


%%% Local Variables:
%%% mode: latex
%%% TeX-master: "main"
%%% End:
