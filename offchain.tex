\section{Off-Chain Protocol}\label{sec:offchain}

This section describes the actual Coordinated Hydra Head protocol, an even more simplified version of the original publication \cite{hydrahead20}. See the protocol overview in Section \ref{sec:overview} for an introduction and notable changes to the original protocol.

The protocol is specified as a reactive system that processes three kinds of events:
\begin{enumerate}
  \item on-chain protocol transactions introduced in the previous section (\ref{sec:on-chain}),
  \item off-chain network messages sent between protocol actors (parties), and
  \item client commands as received from the environment.
\end{enumerate}

The behavior is fully specified with corresponding sections in
Figure~\ref{fig:head_coordinated} using variables and notation introduced in the
following paragraphs.

\subsection{Assumptions}
\begin{itemize}
  \item every network event received from a specific party is checked for
        (channel) authentication. It is the implementer’s duty to find a
        suitable authentication process for the communication channels.
  \item The head protocol gets correctly (and with completeness) notified about
        observed on-chain transactions belonging to the respective head
        instance.
  \item The clients pre-agreed on the head-instance parameters ahead of time:
        public keys, etc (they initialize their “expectations” with the
        OnClient-Init event below)
  \item All events are processed to completion (run-to-completion semantics) and there is no pre-emption.
  \item Given the specification, events may pile up forever; implementations
        might want to consider these situations (i.e. potential for DoS); we
        note though, that from a security standpoint, these situations are
        identical to a non-collaborative peer
  \item The lifecycle of the Head on-chain does not cross Hard Fork or Protocol
        Parameters Update boundaries. Note these events are announced in advance
        hence it should be possible for implementations to react in such a way
        as to expedite closing of the Head before HF or PParamsUpdate
\end{itemize}

\subsection{Notation}
\begin{itemize}
  \item ‘check P’ means that the following code is only executed if P holds and
        otherwise skipped
  \item ‘wait P’ is a blocking wait for condition P (we always use this in a way
        that implies dishonest participation if it does not terminate)
  \item All code blocks are assumed to be atomic
  \item ‘fail’ implies detection of cheating (i.e., head can be closed), no
        further requests or acknowledgments must be sent.
\end{itemize}



\subsection{Variables}

\begin{center}
\begin{tabular}{|l|l|}\hline
  $\hats$  & Sequence number of seen snapshot. \\ \hline
  $\bars$  & Sequence number of confirmed snapshot. \\ \hline
  $\hatmU$ & Seen snapshot object. \\ \hline
  $\barmU$ & Confirmed snapshot object. \\ \hline
  $\hatmL$ & Local ledger extension using seen transactions to prepare the next snapshot.\\ \hline
  $\mT$    & Set of all transactions ever received via reqTx (independent of their validity or mutual conflicts).\\  \hline
  $\hatmT$ & Set of transactions that extend the $\barmU$ (or $\hatmU$) to form $\hatmL$. \\
           & Transaction candidates that go into the next snapshot (if this party is the next leader).\\ \hline
  $\barmT$ & Set of all confirmed transactions up to the latest confirmed snapshot.\\  \hline
\end{tabular}
\end{center}

In particular, note, that the original variables $\hatmT$ and $\barmT$ are not required anymore because they
were used in context of hanging transactions. We thus reused them to fit the new context.

\begin{figure*}[t!]

  \def\sfact{0.8}
  \centering
  \begin{algobox}{Coordinated Hydra Head}
    \medskip
    \begin{tabular}{c}
      \begin{tabular}{cc}
        \adjustbox{valign=t,scale=\sfact}{
         \begin{walgo}{0.65}
          %%% INIT
          \On{$(\hpInit,i,\hpPuv,\hpPr,\Uinit)$ %
            from client}{ %
            % let $i$ be position of $\hpPr$'s public key in $\hpPuv$
            % \; %
            $\msVKL \gets \hpPuv$ \; %
            $\msCVK \gets \msCombVK(\msVKL)$ \; %
            $\msSK \gets \hpPr$ \; %c

            $\hats,\bars \gets 0$ \; %
            $\hatmU, \barmU \gets \Sno (0,\Uinit)$ \; %
            % $\hatmL,\barmL \gets \Uinit$ \; %
            $\hatmL \gets \Uinit$ \; %
            % $\hatmT,\barmT \gets \emptyset$ \; %
            $\mT, \hatmT,  \barmT \gets \emptyset$ \; % all, seenforsnap, confirmedforsnap
          }

          
          \vspace{12pt} %
          
          %%% NEW TX
          \On{$(\hpNew,\tx)$ %
            from client}{ %
            \Req $\validTx(\tx)$ and %
            $\hatmL \applytx \tx \neq \bot$ \; %

            % $\hash \gets \hashF(\tx)$ \; %

            % $\hySignedT[\hash] \gets \hyTxo(i,\tx)$ \; %
            
            \Multi $(\hpRT,\tx)$ \; %
          }           

        \end{walgo}
        }
        &

        \adjustbox{valign=t,scale=\sfact}{
        \begin{walgo}{0.6}
          %%% NEW SN  
          \On{$(\hpNS)$ %
            from client}{ // assumed to be called only once%%
              
            \Req $\hpLdr(\bars + 1) = i$ %
            and $\hats = \bars$ \;
             
            % $\hySignedU \gets %
            % \hySto(\hyConfSN + 1,\hyConfL,\hyConfT)$ \; %

            %$\hatmT \gets 
            
            \Multi $(\hpRS,\bars+1,\hatmT\hpProjH)$ \; %
          }

          \vspace{12pt} %

          %%% CLOSE
          \On{$(\hpClose)$ from client}{ %
            \Return $(\barmU.U, \barmU.s, \barmU.\msCSig)$ \; %
          }
          
          \vspace{12pt} %
          
          %%% CONTEST
          \On{$(\hpCont,\eta)$ from client}{ %
            $(U_\eta,s_\eta) \gets \eta$ \; %

            \Req $s_\eta > \bars$ %
             
            \Return $(U_\eta,s_\eta)$ \; %
          }

          \vspace{12pt}

          % \On{$(\hpFinalize,\eta)$ from client}{ %
          %   $(\hash,s,T) \gets \eta$ \; %            
          % }    
          

        \end{walgo}
          }
        % &
        
        % \adjustbox{valign=t,scale=\sfact}{
        % \begin{walgo}{0.3}
        %   \If{bla}{
        %     bla \; %
        %   }      
        % \end{walgo}} %

      \end{tabular}
      
      \vspace{-10pt}
      \\
      % \ifdefined\IsLLNCS %
        \multicolumn{1}{l}{\line(1,0){490}} %
      % \else %
      %   \multicolumn{1}{l}{\line(1,0){465}} %
      % \fi %
      \\

      \begin{tabular}{c@{}c}
        \adjustbox{valign=t,scale=\sfact}{
        \begin{walgo}{0.65}
          %%% REQ TX
          \On{$(\hpRT,\tx)$ %
            from $\party_j$}{ %
            \Req $\validTx(\tx)$ and %
            $\hatmL \applytx \tx \neq \bot$ \; %
            
            $\mT[\hash] \gets \tx$ % all seen txs

            % \If{$(\hats\in\{\bars,\bars+1\}) \wedge (\hpLdr(\hats+1) = i) \wedge (\hatmL\applytx\tx\not=\bot)$}{
              $\hatmT[\hash] \gets \mT[\hash]$ % candidates for next snapshot \; %

              $\hatmL \gets \hatmL\applytx\tx$ %

            \If{
                \Req $\hpLdr(\bars + 1) = i$ %
                and $\hats = \bars$ \;
                \Multi $(\hpRS,\bars+1,\hatmT\hpProjH)$ \; %
                }
            
          }

          \vspace{12pt} %

          %%% REQ SN
          \On{$(\hpRS,s,T)$ %
            from $\party_j$}{ %

            \Req $s = \hats + 1$ %
            and $\hpLdr(s) = j$ \; %

            \Wait{$\bars = \hats$ %
              and $\barmU\applytx T \not= \bot$ % meaning reach is 'complete' for extension
            }{ %
               $\hats \gets \bars + 1$ \; %

               $\forall\tx\in\Reach^{\mT}(T)$:
                \Out $(\hpSeen,\tx)$ \; %
              
               $\hatmU \gets
               \Sno(\hats,\barmU\applytx T )$ \; %

               $\msSig_i \gets %
               \msSign(\hyPr, (\hats \| U_{0}.h) \| \hatmU.h)$ \; %
 
               \Send $(\hpAS,\hats,\msSig_i)$ %
               to $\party_j$ \; %

               % \If{$\hpLdr(\hats+1) = i$}{
                 $\hatmT :\subseteq_{\mbox{max}} \mT$ s.t. $\hatmU\applytx\hatmT\not=\bot$ \; %
                 $\hatmL \gets \hatmU\applytx\hatmT$
               % }
            }
           }
          
        \end{walgo}
        }
        &

        \adjustbox{valign=t,scale=\sfact}{
        \begin{walgo}{0.6}
          %%% ACK SN
          \On{$(\hpAS,s,\msSig_j)$ %
            from $\party_j$}{ %

            \Req $s \in \{\hats,\hats+1\}$ \; % 2-round ack/confirmation
            % and $\hySignedU.\hySN = \hySN$ \; %

            \Wait{$\hats=s$
            }{ %
            
            \Req $\hatmU.\hpSigs[j] = \undefined$ \; %

            $\hatmU.\hpSigs[j] \gets \msSig_j$ \; %

            \If{$\forall k: \hatmU.\hpSigs[k] \neq \undefined$}{ %
              $\msCSig %
              \gets \msComb((s \| U_{0}.h) \| \hatmU.\hash, %
              \msVKL, \hatmU.\hpSigs)$ \; %
              
              \If{$\msVfy(\msCVK, (\hats \| U_{0}.h) \| \hatmU.\hash,\msCSig)$}{ %
              $\bars \gets s$ \; %
              $\hatmU.\msCSig \gets \msCSig$ \; %
              $\barmU \gets \hatmU$ \; %
              $\barmT \gets \barmT \cup \barmU.T$ \; %

              $\forall \tx\in T$: \Out $(\hpConf,\tx)$ \; %
  
            }\Else{
             \fail
            }%
            }
           }
          }  
        \end{walgo}

          }
          
        % &
        
        % \adjustbox{valign=t,scale=\sfact}{
        % \begin{walgo}{0.3}
        %   \If{bla}{
        %     bla \; %
        %   }      
        % \end{walgo}} %

      \end{tabular}


    \end{tabular}
    \bigskip
  \end{algobox}
  
  \caption{Head-protocol machine for the \emph{coordinated head} from the perspective of party
    $\party_i$.}
  \label{fig:head_coordinated}

\end{figure*}



%%% Local Variables:
%%% mode: latex
%%% TeX-master: "main"
%%% End:
[h]

Comments: 
\begin{itemize}
\item Fix: from client or from network: same function called twice should trigger at most one wait.
  For instance, $\hpRS$ called twice must not queue the 'wait' part for multiple execution.
\item Fix: $\hpNS$ called twice must not trigger local $\hpRS$ twice.
\item Add: garbage collection ($\hatmT,\barmT$,etc.)
\item Extensive code explanation (per line of code)
\end{itemize}

The following figure visualizes the possible event flows from the view of a single party.
\begin{figure}[ht]
  \centering
  \includegraphics[width=\linewidth*2/3]%
                  {fig/reqackstates.png}
                  \caption{State diagram of reqSn/ackSn cycles.}
                  \label{fig:sim:baselines}
\end{figure}

%%% Local Variables:
%%% mode: latex
%%% TeX-master: "main"
%%% End:
