\section{On-chain Protocol}\label{sec:mainchain}

We describe the details of the \emph{on-chain} protocol controlling a
Hydra head (see Fig.~\ref{fig:SM_states_basic}) using the CEM abstraction \&
notation (see Section~\ref{sec:cem}). In addition of standard CEM modeling, we
also provide the formal conditions $\cemTxCon$ which a transition need to
satisfy and also include them in the accompanying text.

The following sections describe the structure of each of the transactions comprising 
the Head protocol: Initial, Commit, Abort, CollectCom, Close, Contest, and FanOut. 
Following the eUTxO model, this structure is enforced on-chain through \emph{validators}, eg. scripts attached to each UTxO run as part of the ledger's validation. 

\todo{explain how scripts/contracts relate to this}

% TODO: no generic OCV anymore
% \dparagraph{Onchain verification algorithms.} The status of the head
% is maintained in a variable $\eta$, which is part of the SM state and
% updated by so-called \emph{onchain verification (OCV) algorithms}
% $\ocvInitial$, $\ocvClose$, $\ocvContest$, and $\ocvFinal$. In the context
% of the mainchain protocol, these OCV algorithms are intentionally kept
% as generic as possible; this keeps the mainchain SM compatible with
% many potential head-protocol variants.
% The concrete OCV algorithms for the head protocol specified in this
% paper are given in context of the head protocol itself as they depend
% on the specific head-protocol internals: verification of head-protocol
% certificates and related onchain state updates.
% As such, the OCV algorithms can be seen as abstract mainchain algorithms
% implemented by the specific head protocol. Consequently, the OCV
% implementation for our head protocol is described in Section~\ref{sec:hpocv}.

\subsection{Initial transaction} 

The \mtxInit{} transaction (see
Fig.~\ref{fig:SM_commit_tx}) establishes the initial state of the protocol as the tuple
$$
(\stInitial,\hpAK,\hppuv,\nop,\cPer),
$$ 
where:
\begin{menumerate}
    \item $\stInitial$ is
a state identifier, 
   \item $\hpAK$ is the aggregated multi-signature key established
during the setup phase, 
  \item $\hppuv$ is the list of all participants verification
keys $(k_1,\ldots,k_\nop)$ exchanged during the setup phase and identifying the
head members, 
  \item $\nop$ is the number of head members, and 
  \item $\cPer$ is the length of the contestation period.
\end{menumerate} 

\subsubsection{Head Tokens}

The \mtxInit{} transaction also mints tokens whose \emph{Minting Policy} is $\mathsf{cid}$, defined as 
$$
\mathsf{cid} = H(\muHead(\mathsf{tx_{in}})),
$$
where $\mathsf{tx_{in}}$ is one of the inputs spent in the \mtxInit{} transaction. From the ledger preventing double-spending and the uniqueness of $\mathsf{tx_{in}}$, $\mathsf{cid}$ is guaranteed to be unique and can be used to identify the newly initialized head. 

Two kinds of tokens are minted:
\begin{itemize}
\item $\mathsf{ST}$: A single \emph{State Thread} token marking the output carrying the state of the protocol on-chain, whose name is the well known string \texttt{HydraHeadV1},
\item $\mathsf{PT}_i,\ i\in \{1 \dots \nop \}$: One participation token per participant, where the token name is the participant's verification key hash $H(k_i)$. This will be used to authenticate participants in protocol transactions.
\end{itemize}

\subsubsection{Initial Outputs}

The \mtxInit{} transaction has $\nop$ outputs, where each output is
locked by validator $\nuInitial$ and the $\ith i$ output has the participation
token $\mathsf{PT}_i$ in its value, while the datum is unused. Validator $\nuInitial$
ensures that the output is consumed by either by \mtxAbort{} (see section \ref{sec:abort-tx} below) or \mtxCom{} (see section \ref{sec:commit-tx} below).

The general well-formedness and validity of the \mtxInit{} transaction is
checked on the mainchain. The head members can additionally check whether the head
parameters match the parameters agreed on during the setup phase.

\subsection{Commit Transaction}\label{sec:commit-tx}

\begin{figure}[h]

  \centering

  % \includegraphics[width=\textwidth/2,trim=130 330 430 50,clip]{figures/SM_commit_tx.pdf}

  % TODO: clean draw marked up version
  \includegraphics[width=\textwidth*2/3]{figures/SM_commit_tx.png}

  \caption{
    \mtxInit{} transaction (left) with one \mtxCom{} transaction
    (right) attached locking one output (center).}\label{fig:SM_commit_tx}

\end{figure}


%%% Local Variables:
%%% mode: latex
%%% TeX-master: "main"
%%% End:


A \mtxCom{} transaction consumes one $\nuInitial$ UTxO and zero or one \emph{committed} UTxO, and has one output locked by validator $\nuCom$. 

The $\nuInitial$ validator ensures the transaction has the following structure:
\begin{menumerate}
    \item the transaction is signed with verification key corresponding to the asset name in $\mathsf{PT}_i$,
    \item the redeemer for $\nuInitial$ is referencing the output to commit $ \rho = \txOutRef_{commit}$, with output $o_{commit}$ holding value $\val_{commit}$
    \item the committed value is in the output $\val' = \{\mathsf{PT}_{i}\} \cup \val_{commit} $,
    \item the data field $\delta$ of the output locked by $\nuCom$ includes 
    $U_i = \recordUTxO(\txOutRef_{commit},o_{commit})$, where 
    $$\recordUTxO (\txOutRef, o) = (\txOutRef, \bits(o)),
    $$
    \todo{removed the part about the cid being  in the datum as it's not what we implement and it 
    does not make sense with how we say we construct the output earlier.}
    \item no minting or burning happens.
\end{menumerate}

The $\nuCom$ validator ensures the output is collected by either a \mtxCCom{} or \mtxAbort{} transaction.
\begin{boxM}
The \mtxCom{} transaction does not take part in the state machine logic.
\end{boxM}


\subsection{CollectCom Transaction} 

\begin{figure}[t!]

  \centering

  %\includegraphics[width=\textwidth/2-2em,trim=420 80 120 300,
  %clip]{figures/SM_initial_open.pdf}

  \includegraphics[width=\textwidth/2]{figures/SM_initial_open.pdf}

  \caption{\mtxInit{} transaction (left) with \mtxCCom{} transaction
    (right) and \mtxCom{} transactions (center).}
  \label{fig:SM_initial_open}

\end{figure}



%%% Local Variables:
%%% mode: latex
%%% TeX-master: "main"
%%% End:


The \mtxCCom{} transaction collects all outputs from \mtxCom{} transactions participating in the same head and advances the state of the CEM state machine:


\begin{menumerate}
  \item committed UTxOs (datums) are recorded correctly in
  $\eta = \mathsf{Combine}(U_{1}, \ldots, U_{n})$,
  \item all committed value captured and no additional funds ``enter'' or ``leave''
  $\val' = \bigcup_{i=1}^{n} p_{i} \cup \val_{i}$,
  \item all tokens present in output
  $|\{cid \rightarrow . \rightarrow 1\} ~ \mathsf{in} ~ \val'| = \nop + 1$
  \todo{really count?},
  \item the transaction is signed by a verification key $k' \in \hppuv$ with its
  hash corresponding to the asset name of one of the participation tokens
  $\{p_1 \dots p_n\}$,
  \item unchanged parameters $\mathsf{cid}$, $\hpAK$, $\hppuv$, $\nop$, and
  $\cPer$ in the data field,
  \item no minting or burning happens.
\end{menumerate}

Each of the $\nuCom_i$ validators, for $i \in \{ 1\dots n\}$, checks: 
    \begin{menumerate}
      \item the ST token is present in the output value $\{cid \rightarrow ST \rightarrow 1\} \subseteq \val'$
      \item with the correct $\mathsf{cid}$ in the datum $(cid,.) = \delta$.
    \end{menumerate}


All parameters $\hpAK$, $\hppuv$, $\nop$, and $\cPer$ remain part of the state,
but in addition, a value $\eta \gets \ocvInitial(U_1,\ldots,U_n)$ is stored in
the state. The idea is that $\eta$ stores information about the initial UTxO
set, which is made up of the individual UTxO sets $U_i$ collected from the
commit transactions, in order to verify head-status information later (see
below).

It is also required that all $\nop$ participation tokens be present in the SM
output of the \mtxCollect{} transaction. This ensures that the \mtxCollect{}
transaction collects all $\nop$ commit transactions, because $\nuInitial$ does
not allow a \mtxCollect{} transaction to consume the outputs of the \mtxInit{}
transaction directly. The only way to post the \mtxCollect{} transaction is if
each head member has posted a \mtxCommit{} transaction.


\todo{WIP after here}

\subsection{Abort Transaction}\label{sec:abort-tx} 

The \mtxAbort{} transaction
(see Fig.~\ref{fig:SM_initial_final}) allows a party to abort the
creation of a head in case some parties fail to post a commit
transaction.  The final state does not contain any information (beyond
its identifier), but it is ensured that (1) the outputs $U$ correspond
to the union of all committed UTxO sets $U_i$ and (2) all
participation tokens are burned.

Each of the $\nuCom_i$ validators, for $i \in \{ 1\dots n\}$, checks: 

\begin{menumerate}
  \item the ST token is getting burned $\{cid \rightarrow ST \rightarrow -1\} \subseteq \mathsf{Mint}$
  \item with the correct $\mathsf{cid}$ in the datum $(cid,.) = \delta$.
\end{menumerate}


\begin{figure}

  \centering

  % \includegraphics[width=\textwidth/2-2em,trim=350 20 240 300,
  % clip]{figures/SM_initial_final.pdf}
  % TODO: clean draw marked up version
  \includegraphics[width=\textwidth/2]{figures/SM_initial_final.png}

  \caption{\mtxInit{} transaction (left) with \mtxAbort{} transaction (right)
    and \mtxCom{} transactions (center).}\label{fig:SM_initial_final}

\end{figure}



%%% Local Variables:
%%% mode: latex
%%% TeX-master: "main"
%%% End:



\subsection{Close Transaction} 

In order to close a head, a head
member may post the \mtxClose{} transaction (see
Fig.~\ref{fig:SM_open_closed}), which results in a SM transition
from the $\stOpen$ state to the $\stClosed$ state.\newline

Description of performed validator checks:

\begin{enumerate}
  \item Verify off-chain signatures on a snapshot or keep recorded UTxO data unchanged if there is no snapshot 
  \item Ensure that the close transaction validity is bounded by contestation period: $Tmax - Tmin <= CP$  
  \item Make sure deadline $DL$ is $CP$ time added to the close transaction upper bound: $DL' = Tmax + CP$
  \item Record contestation period in the datum 
  \item Check that close transaction is signed by one of the valid participants
\end{enumerate}
\newline


Once a \mtxClose{} transaction has been posted, a \emph{contestation period}
begins which should last at least $\cPer$ slots.  Hence, the last slot
$\Tfinal$ of the contestation period is recorded in the state, and it
is ensured that $\Tfinal \geq \txRmax' + \cPer$.

Finally, the SM state is extended by a set $\contesters$
initialized to the poster's signing key, i.e.,
$\contesters \gets \{k'\}$.   $\contesters$ is used to ensure that no party
posts more than once during the contestation period.


\begin{figure}

  \centering

  % \includegraphics[width=\textwidth/2]{figures/SM_open_closed.pdf}
  % TODO: clean draw marked up version
  \includegraphics[width=\textwidth/2]{figures/SM_open_closed.png}

  \caption{\mtxCCom{} transaction (left) with \mtxClose{} transaction
    (right).}\label{fig:SM_open_closed}

\end{figure}



%%% Local Variables:
%%% mode: latex
%%% TeX-master: "main"
%%% End:



\subsection{Contest Transaction} 

If the party first closing a head posts
outdated/incomplete information about the current state of the head,
any other party may post a \mtxContest{} transaction (see
Fig.~\ref{fig:SM_closed_closed}), which causes a state transition from
the $\stClosed$ state to itself.  The transition handles update
information $\xi$ by passing it through OCV algorithm $\ocvContest$,
resulting in a new OCV status
$\eta' \gets \ocvContest (\hpAK,\eta,\xi)$.  OCV algorithm
$\ocvContest$ uses the previous OCV status $\eta$ and $\hpAK$ to check
the update information $\xi$.  Similarly to $\ocvClose$, $\ocvContest$
may output $\bot$, but in order for a \mtxContest{} transaction to be
valid $\eta' \neq \bot$ is required.

The \mtxContest{} transaction is only valid if the old set
$\contesters$ of parties who have contested (or closed) so far does not yet
include the poster, i.e., $k' \notin \contesters$.  If this check
passes, the set is extended to include the poster of the \mtxContest{}
transaction, i.e., $\contesters' \gets \contesters \cup \{k'\}$.
Furthermore, \mtxContest{} transactions may only be posted up until
$\Tfinal$, i.e., it is required that $\txRmax' \leq \Tfinal$.

Observe that during the contestation period, up to $\nop-1$
\mtxContest{} transactions may be posted (of course, the parameter
$\cPer$ has to be chosen large enough as to allow each head member to
potentially post a \mtxClose{}/\mtxContest{} transaction).


\begin{figure}

  \centering

  %\includegraphics[width=\textwidth/2-2em,trim=280 120 160 260,
  %clip]{figures/SM_closed_closed.pdf}

  \includegraphics[width=\textwidth/2]{figures/SM_closed_closed.pdf}

  \caption{\mtxClose{}/\mtxContest{} transaction (left);
    \mtxContest{} transaction (right)}
  \label{fig:SM_closed_closed}

\end{figure}



%%% Local Variables:
%%% mode: latex
%%% TeX-master: "main"
%%% End:


\begin{figure}[t!]

  \centering

  %\includegraphics[width=\textwidth/2-2em,trim=250 40 240 280,
  %clip]{figures/SM_closed_final.pdf}

  \includegraphics[width=\textwidth/2]{figures/SM_closed_final.pdf}

  \caption{\mtxClose{}/\mtxContest{} transaction (left);
    \mtxFanout{} transaction (right)}
  \label{fig:SM_closed_final}

\end{figure}



%%% Local Variables:
%%% mode: latex
%%% TeX-master: "main"
%%% End:


\subsection{Fan-Out Transaction}  

Once the contestation phase is over, a head
may be finalized by posting a \mtxFanout{} transaction, taking the SM
from $\stClosed$ to $\stFinal$.  The \mtxFanout{} transaction must
have outputs that correspond to the most recent head state.  To that
end, OCV predicate $\ocvFinal$ checks the transaction's output set $U$
against the information recorded in $\eta$.  The \mtxFanout{}
transaction is only valid if $\ocvFinal$ outputs $\true$.  Moreover,
to ensure that the \mtxFanout{} transaction is not posted too early,
$\txRmin' > \Tfinal$ is required.  Finally, all participation tokens
must be burned.


%%% Local Variables:
%%% mode: latex
%%% TeX-master: "main"
%%% End:
