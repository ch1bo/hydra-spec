\section{On-chain Protocol}\label{sec:mainchain}

We describe the details of the \emph{on-chain} protocol controlling a
Hydra head (see Fig.~\ref{fig:SM_states_basic}) using the CEM abstraction \&
notation (see Section~\ref{sec:cem}). In addition of standard CEM modeling, we
also provide the formal conditions $\cemTxCon$ which a transition need to
satisfy and also include them in the accompanying text.

The following sections describe the structure of each of the transactions comprising the Head protocol: Initial, Commit, Abort, CollectCom, Close, Contest, and FanOut. Following the eUTxO model, this structure is enforced on-chain through \emph{validators} which are \emph{scripts instances} attached to each UTxO and run as part of the ledger's validation of the transaction. The protocol defines three different scripts: 
\begin{itemize}
    \item $\nuHead$ is the script controlling the protocol state-machine logic,
    \item $\nuInitial$ is the script used to control the initial distribution of tokens to Head participants, and
    \item $\nuCom$ controls the collecting of \emph{committed} UTxO for use in the Head. 
\end{itemize}

% TODO: no generic OCV anymore
% \dparagraph{Onchain verification algorithms.} The status of the head
% is maintained in a variable $\eta$, which is part of the SM state and
% updated by so-called \emph{onchain verification (OCV) algorithms}
% $\ocvInitial$, $\ocvClose$, $\ocvContest$, and $\ocvFinal$. In the context
% of the mainchain protocol, these OCV algorithms are intentionally kept
% as generic as possible; this keeps the mainchain SM compatible with
% many potential head-protocol variants.
% The concrete OCV algorithms for the head protocol specified in this
% paper are given in context of the head protocol itself as they depend
% on the specific head-protocol internals: verification of head-protocol
% certificates and related onchain state updates.
% As such, the OCV algorithms can be seen as abstract mainchain algorithms
% implemented by the specific head protocol. Consequently, the OCV
% implementation for our head protocol is described in Section~\ref{sec:hpocv}.

\subsection{Initial transaction} 

The \mtxInit{} transaction (see
Fig.~\ref{fig:SM_commit_tx}) establishes the initial state of the protocol as the tuple
$$
(\stInitial,\hpAK,\hppuv,\nop,\cPer)
$$ 
where:
\begin{menumerate}
    \item $\stInitial$ is
a state identifier, 
   \item $\hpAK$ is the aggregated multi-signature key established
during the setup phase, 
  \item $\hppuv$ is the list of all participants verification
keys $(k_1,\ldots,k_\nop)$ exchanged during the setup phase and identifying the
head members, 
  \item $\nop$ is the number of head members, and 
  \item $\cPer$ is the length of the contestation period.
\end{menumerate} 

\subsubsection{Head Tokens}

The \mtxInit{} transaction also mints tokens whose \emph{Minting Policy} is $\mathsf{cid}$, defined as 
$$
\mathsf{cid} = \mathbb{H}(\muHead(\mathsf{tx_{in}})),
$$
where $\mathsf{tx_{in}}$ is one of the inputs spent in the \mtxInit{} transaction. From the ledger preventing double-spending and the uniqueness of $\mathsf{tx_{in}}$, $\mathsf{cid}$ is guaranteed to be unique and can be used to identify the newly initialized head. \\
\\
Two kinds of tokens are minted:
\begin{itemize}
\item $\mathsf{ST}$: A single \emph{State Thread} token marking the output carrying the state of the protocol on-chain, whose name is the well known string \texttt{HydraHeadV1},
\item $\mathsf{PT}_i,\ i\in \{1 \dots \nop \}$: One participation token per participant, where the token name is the participant's verification key hash $\mathbb{H}(k_i)$. This will be used to authenticate participants in protocol transactions.
\end{itemize}

\subsubsection{Initial Outputs}

The \mtxInit{} transaction has $\nop$ outputs, where each output is
locked by validator $\nuInitial$ and the $\ith i$ output has the participation
token $\mathsf{PT}_i$ in its value, and $\mathsf{cid}$ as a datum. Validator $\nuInitial$
ensures that the output is consumed by either an \mtxAbort{} (see section \ref{sec:abort-tx} below) or a \mtxCom{} (see section \ref{sec:commit-tx} below) transaction.

The general well-formedness and validity of the \mtxInit{} transaction is
checked on the mainchain. The head members can additionally check whether the head
parameters match the parameters agreed on during the setup phase.

\subsection{Commit Transaction}\label{sec:commit-tx}

\begin{figure}[h]

  \centering

  % \includegraphics[width=\textwidth/2,trim=130 330 430 50,clip]{figures/SM_commit_tx.pdf}

  % TODO: clean draw marked up version
  \includegraphics[width=\textwidth*2/3]{figures/SM_commit_tx.png}

  \caption{
    \mtxInit{} transaction (left) with one \mtxCom{} transaction
    (right) attached locking one output (center).}\label{fig:SM_commit_tx}

\end{figure}


%%% Local Variables:
%%% mode: latex
%%% TeX-master: "main"
%%% End:


A \mtxCom{} transaction consumes one $\nuInitial$ UTxO and zero or one \emph{committed} UTxO, and has one output locked by validator $\nuCom$.

The $\nuInitial$ validator ensures the transaction has the following structure:
\begin{menumerate}
    \item The transaction is signed with a verification key $\mathsf{vk}$ such that 
    $$ 
    \exists (\mathsf{cid} \rightarrow \mathsf{PT}_i \rightarrow 1) \in \val_{commit}, \mathbb{H}(\mathsf{vk}) = \mathsf{PT}_i,
    $$
    \item The redeemer for $\nuInitial$ is referencing the output to commit $ \rho = \txOutRef_{commit}$, with output $o_{commit}$ holding value $\val_{commit},$
    \item The committed value is in the output $\val' = \{(\mathsf{cid} \rightarrow \mathsf{PT}_i \rightarrow 1)\} \cup \val_{commit} $,
    \item The data field $\delta$ of the output locked by $\nuCom$ is $(\mathsf{cid}, U)$ with:
        $$U = (\txOutRef_{commit},\bits(o_{commit})),$$
    \item No minting or burning happens.
\end{menumerate}

\noindent The $\nuCom$ validator ensures the output is collected by either a \mtxCCom{} or \mtxAbort{} transaction.
\begin{boxM}
The \mtxCom{} transaction does not take part in the state machine logic.
\end{boxM}

\subsection{CollectCom Transaction}

\begin{figure}[t!]

  \centering

  %\includegraphics[width=\textwidth/2-2em,trim=420 80 120 300,
  %clip]{figures/SM_initial_open.pdf}

  \includegraphics[width=\textwidth/2]{figures/SM_initial_open.pdf}

  \caption{\mtxInit{} transaction (left) with \mtxCCom{} transaction
    (right) and \mtxCom{} transactions (center).}
  \label{fig:SM_initial_open}

\end{figure}



%%% Local Variables:
%%% mode: latex
%%% TeX-master: "main"
%%% End:


The \mtxCCom{} transaction collects all outputs from \mtxCom{} transactions participating in the same head and advances the state of the CEM state machine:
$$
   (\stInitial,\hpAK,\hppuv,\nop,\cPer) \xrightarrow{\mathsf{collectCom}} (\stOpen,\hpAK,\eta,\hppuv,\nop,\cPer),
$$
where $\eta$ is the hash of the concatenation of the serialised representation of committed outputs:
$$
\eta = \mathbb{H}(\bigoplus_{i=1}^n \mathsf{bits}(U_i)).
$$

The $\nuHead$ validator additionally checks that:
\begin{menumerate}
  \item All committed value captured and no additional funds ``enter'' or ``leave''
  $\val' = \{\mathsf{ST}\} \cup \bigcup_{i=1}^{n} \mathsf{PT}_{i} \cup \val_{i}$,
  \item All tokens present in output
  $|\{cid \rightarrow . \rightarrow 1\} ~ \mathsf{in} ~ \val'| = \nop + 1$,
  \item The transaction is signed by a verification key $k' \in \hppuv$ with its
  hash corresponding to the asset name of one of the participation tokens
  $\{\mathsf{PT}_1 \dots \mathsf{PT}_n\}$,
  \item Unchanged parameters $\mathsf{cid}$, $\hpAK$, $\hppuv$, $\nop$, and
  $\cPer$ in the data field,
  \item No minting or burning happens.
\end{menumerate}

\noindent Each of the $\nuCom^i$ validators, for $i \in \{ 1\dots n\}$, additionally checks:
\begin{menumerate}
    \item The ST token is present in the output value $(cid \rightarrow ST \rightarrow 1) \in \val'$
    \item The $\mathsf{cid}$ in the datum $(cid,.) = \delta$ matches the one from the token.
\end{menumerate}

\subsection{Abort Transaction}\label{sec:abort-tx} 

\begin{figure}

  \centering

  % \includegraphics[width=\textwidth/2-2em,trim=350 20 240 300,
  % clip]{figures/SM_initial_final.pdf}
  % TODO: clean draw marked up version
  \includegraphics[width=\textwidth/2]{figures/SM_initial_final.png}

  \caption{\mtxInit{} transaction (left) with \mtxAbort{} transaction (right)
    and \mtxCom{} transactions (center).}\label{fig:SM_initial_final}

\end{figure}



%%% Local Variables:
%%% mode: latex
%%% TeX-master: "main"
%%% End:



The \mtxAbort{} transaction
(see Fig.~\ref{fig:SM_initial_final}) allows a party to abort the
creation of a head.  The state is transitioned to $\mathsf{final}$:

$$
   (\stInitial,\hpAK,\hppuv,\nop,\cPer) \xrightarrow{\mathsf{abort}} \stFinal.
$$

\noindent The $\nuHead$ validator ensures that:
\begin{menumerate}
 \item All outputs committed into the head are recreated as is: $\forall i \in \{1\dots\nop\}, \mathbb{H}(O[i]) = \delta_i$,
 \item All participation tokens are burnt: $\forall i \in \{1\dots\nop\}, \{\mathsf{cid} \rightarrow \mathsf{PT}_i \rightarrow -1\} \subseteq \mathsf{Mint}.$
\end{menumerate} 

\noindent For each of the $\nuInitial$ validators consumed, checks:
\begin{menumerate}
  \item The data field $\delta$ of the output locked is $\mathsf{cid}$,
  \item The ST token is getting burned $\{cid \rightarrow ST \rightarrow -1\} \subseteq \mathsf{Mint}.$ 
\end{menumerate}

\noindent For each of the $\nuCom$ validators consumed, checks:
\begin{menumerate}
  \item The ST token is getting burned $\{cid \rightarrow ST \rightarrow -1\} \subseteq \mathsf{Mint}$
  \item The correct $\mathsf{cid}$ is in the datum $(cid,.) = \delta$.
\end{menumerate}

\subsection{Close Transaction} 

\begin{figure}

  \centering

  % \includegraphics[width=\textwidth/2]{figures/SM_open_closed.pdf}
  % TODO: clean draw marked up version
  \includegraphics[width=\textwidth/2]{figures/SM_open_closed.png}

  \caption{\mtxCCom{} transaction (left) with \mtxClose{} transaction
    (right).}\label{fig:SM_open_closed}

\end{figure}



%%% Local Variables:
%%% mode: latex
%%% TeX-master: "main"
%%% End:


In order to close a head, a head member may post the \mtxClose{} transaction (see Fig.~\ref{fig:SM_open_closed}), which results in a SM transition
from the $\stOpen$ state to the $\stClosed$ state:
$$
(\stOpen,\hpAK,\eta,\hppuv,\nop,\cPer) \xrightarrow[\xi]{\mathsf{close}} (\stClosed,\hpAK,\eta',\eta_0,\mathcal{C},\hppuv,\nop,\cPer,\Tfinal) 
$$
with $\eta_0 = \eta$.

The $\nuHead$ validator perform those additional checks:
\begin{enumerate}
  \item Signer is one of the participants, $k' \in \hppuv,$
  \item $\xi$ is a valid signature of $\eta'$ w.r.t. to $\hpAK$,
  \item Recorded snapshot state is consistent:
    $$
    \eta' = \left\{\begin{array}{ll}
         (s, \mathbb{H}(U')), & \mathrm{if}\ s > 0,\\
         (0, \eta_0) & \mathit{otherwise} 
    \end{array}\right.,
    $$
  \item Record contestation deadline, $\Tfinal = T_{\mathsf{max}} + T$,
  \item Ensure timeliness of the transaction $T_{max} - T_{min}\leq T$ 
  \item Record public key hash of participant, $\mathcal{C} = \{k'\},$
  \item Value is preserved, $val' = val,$
  \item Nothing minted or burned.
\end{enumerate}

\subsection{Contest Transaction} 

\begin{figure}

  \centering

  %\includegraphics[width=\textwidth/2-2em,trim=280 120 160 260,
  %clip]{figures/SM_closed_closed.pdf}

  \includegraphics[width=\textwidth/2]{figures/SM_closed_closed.pdf}

  \caption{\mtxClose{}/\mtxContest{} transaction (left);
    \mtxContest{} transaction (right)}
  \label{fig:SM_closed_closed}

\end{figure}



%%% Local Variables:
%%% mode: latex
%%% TeX-master: "main"
%%% End:


The \mtxContest{} transaction (see
Fig.~\ref{fig:SM_closed_closed}) is posted by a party to prove the currently $\stClosed$ state is not the latest one. This causes the following transition in the CEM state machine (eg. change datum of output locked by $\nuHead$ script):
$$
   (\stClosed,\hpAK,\eta,\eta_0,\mathcal{C},\hppuv,\nop,\cPer,\Tfinal) \xrightarrow[\xi]{\mathsf{contest}} (\stClosed,\hpAK,\eta',\eta_0,\mathcal{C}', \hppuv,\nop,\cPer,\Tfinal'),
$$
with $\eta = (s, \mathbb{H}(U))$ and $\eta' = (s', \mathbb{H}(U')).$

\noindent The $\nuHead$ validator additionally checks that:
\begin{menumerate}
  \item Value is preserved $\val' = \val$,
  \item The transaction is signed by a verification key $k' \in \hppuv$ such that 
  $\exists (\mathsf{cid} \rightarrow \mathsf{PT}_i \rightarrow 1) \in \mathsf{val}, \mathsf{PT}_i = H(k'),$
  \item The signer has not already contested $k' \not\in \mathcal{C},$  and it's added to the signers set: $\mathcal{C}' = \mathcal{C} \cup k',$
  \item $\xi$ is a valid signature of $(\eta_0 || \eta')$ with respect to $\hpAK$, 
  \item $s' > s$, 
  \item Transaction is posted before deadline: $T_{\mathsf{max}} \leq \Tfinal$,
  \item Contestation deadline is updated:
     $$
     \Tfinal' = 
        \left\{\begin{array}{ll}
             \Tfinal,     &  |\mathcal{C}'| = n, \\
             \Tfinal + T, &  \mathit{otherwise} \\
        \end{array}\right.,
     $$
  \item No minting or burning happens.
\end{menumerate}


\subsection{Fan-Out Transaction}  

\begin{figure}[t!]

  \centering

  %\includegraphics[width=\textwidth/2-2em,trim=250 40 240 280,
  %clip]{figures/SM_closed_final.pdf}

  \includegraphics[width=\textwidth/2]{figures/SM_closed_final.pdf}

  \caption{\mtxClose{}/\mtxContest{} transaction (left);
    \mtxFanout{} transaction (right)}
  \label{fig:SM_closed_final}

\end{figure}



%%% Local Variables:
%%% mode: latex
%%% TeX-master: "main"
%%% End:


Once the contestation phase is over, a head
may be finalized by posting a \mtxFanout{} transaction, taking the SM
from $\stClosed$ to $\stFinal$.  

$$
   (\stClosed,\hpAK,\eta,\eta_0,\mathcal{C},\hppuv,\nop,\cPer,\Tfinal) \xrightarrow{\mathsf{fanout}} \stFinal,
$$
with $\eta = (s, H)$ and $\eta_0 = (0, H_0).$
The $\nuHead$ validator performs the following checks:
\begin{enumerate}
  \item Correct outputs are created: 
  $$
  \mathcal{H}(\bigoplus_{i=1}^n \mathsf{bits}(O[i])) = 
    \left\{
    \begin{array}{ll}
        H_0, & s = 0 \\
        H, &\mathit{otherwise}
    \end{array}
    \right.,
  $$
  \item All tokens are burnt: 
     $\mathsf{ST} \cup \{(\mathsf{cid} \rightarrow \mathsf{PT}_i \rightarrow -1) \mid i \in \{1\dots\nop\}\} \subseteq \mathsf{Mint},$
  \item Transaction is posted after contestation deadline $\txRmin' > \Tfinal.$
\end{enumerate}

%%% Local Variables:
%%% mode: latex
%%% TeX-master: "main"
%%% End:
