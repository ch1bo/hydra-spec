\section{Protocol Overview}
\label{sec:overview}

TODO: Apply marked up changes from remarkable

The Hydra protocol provides functionality to lock a set of UTxOs on a
blockchain, referred to as the \emph{mainchain}, and evolve it inside a so-called offchain \emph{head}, independently of the
mainchain.  At any point, the head can be closed with the effect that the
locked set of UTxOs on the mainchain is replaced by the latest set of UTxOs
inside the head.
The protocol guarantees full wealth preservation: no generation
of funds can happen offchain, % (i.e., inside a head)
and no responsive
honest party involved in a head can ever lose any funds other than by
consenting to give them away.

The advantage of head evolution from a liveness viewpoint is that,
under good conditions, it can essentially proceed at network speed,
thereby reducing latency and increasing throughput in an optimal
way.  At the same time, the head protocol provides the same smart-contract
capabilities as the mainchain.

To avoid overloading with technical details, the main body of the paper presents a simplified
version of Hydra to convey the basic concepts and ideas
of the new protocol. Also in the overview, we focus on the simplified protocol
and outline the differences of the full protocol
in Section~\ref{sec:overview_diff}.
A detailed description of the simplified protocol is given in
Sections~\ref{sec:hpsetup}--~\ref{sec:hp},
and Appendix~\ref{sec:hp_cr}. The full protocol is described in
Appendix~\ref{sec:mainchain_full}.


\subsection{The big picture}
\label{sec:overview_bp}

To create a head-protocol instance, any party may take the role of an
\emph{initiator} and ask a set parties (including himself),
the \emph{head members}, to participate in the head by announcing the
identities of the parties.
%This happens in a separate offchain subprotocol.

Each party then establishes pairwise authenticated
channels to all other parties or---if this is not possible---aborts the protocol setup.\footnote{We generally assume that
  mechanisms for establishing pairwise
  authenticated channels are in place, e.g., by
  means of a public-key infrastructure.}

The parties then exchange, via the pairwise authenticated channels,
some public-key material. This public-key material is used both
for the authentication of head-related onchain transactions that
are restricted to head members (e.g., a non-member is not allowed
to close the head) and for multisignature-based event confirmation
in the head.

The initiator then establishes the head by submitting an
\emph{initial} transaction to the mainchain that contains the
head parameters and forges special \emph{participation tokens}
identifying the head members by assigning each token to the public key
distributed by the respective party during the the setup phase.
The initial transaction also initializes a state machine
(see Fig.~\ref{fig:SM_states_basic}) for the head instance that manages
the ``transfer'' of UTxOs between mainchain and head.

Once the initial transaction appears on the mainchain, establishing
the initial state $\stInitial$, each head member can attach a
\mtxCom{} transaction, which locks (on the mainchain) the UTxOs that
the party wants to commit to the head.

The commit transactions are subsequently collected by the \mtxCCom{}
transaction causing a transition from $\stInitial$ to $\stOpen$.  Once
the $\stOpen$ state is confirmed, the head members start running the
offchain \emph{head protocol}, which evolves the initial UTxO set (the
union over all UTxOs committed by all head members) independently of
the mainchain.  For the case where some head members fail to post a
\mtxCom{} transaction, the head can be aborted by going directly from
$\stInitial$ to $\stFinal$.

The head protocol is designed to allow any head member
at any point in time to produce, without interaction, a certificate
for the current head UTxO set.  Using this certificate, the head member
may advance the state machine to the $\stClosed$ state.

Once in $\stClosed$, the state machine grants parties a
\emph{contestation period}, during which each party may (one single
time) contest the closure by providing the certificate for a newer head
UTxO set.  Contesting leads back to the state $\stClosed$.
After the contestation period has elapsed, the state machine may proceed
to the $\stFinal$ state.  The state machine enforces that the
outputs of the transaction leading to $\stFinal$ correspond exactly to
the latest UTxO set seen during the contestation period.


\subsection{The mainchain state machine}
\label{sec:overview_mc}

The mainchain part of the Hydra protocol fulfills two principal
functions: (1) it locks the mainchain UTxOs committed % by head members
to the head while the head is active and (2) it facilitates the
settlement of the final head state back to the mainchain after the
head is closed. In combination, these two functions effectively result
in replacing the initial head UTxO set by the final head UTxO set on
the mainchain in a manner that respects but does not persist the
complete set of head transactions.

\begin{figure}[t!]
  \centering
  \begin{tikzpicture}[>=stealth,auto,node distance=2.8cm, initial text=$\mathsf{init}$, every
    state/.style={text width=10mm, text height=2mm, align=center}]
    \node[state, initial] (initial) {$\stInitial$};
    \node[state] (open) [above right of=initial] {$\stOpen$};
    \node[state] (closed) [right of=open] {$\stClosed$};
    \node[state] (final) [below right of=closed] {$\stFinal$};

    \path[->] (initial) edge [bend left=20] node {$\stCollect$} (open);
    \path[->] (open) edge [bend left=20] node {$\stClose$} (closed);
    \path[->] (closed) edge [bend left=20] node {$\stFanout$} (final);
    \path[->] (closed) edge [loop above] node {$\stContest$} (closed);
    \path[->] (initial) edge node {$\stAbort$} (final);
  \end{tikzpicture}

  \caption{Mainchain state diagram for this version of the Hydra protocol.}\label{fig:SM_states_basic}
\end{figure}

%%% Local Variables:
%%% mode: latex
%%% TeX-master: "main"
%%% End:


The state machine (Fig.~\ref{fig:SM_states_basic}) implementing
the mainchain protocol comprises the four states $\stInitial$,
$\stOpen$, $\stClosed$, and $\stFinal$, where the first two realize
the first function (locking the initial UTxO set) and the
second two realize the second function (settling the final UTxO set on
the mainchain).

State machines inherently sequentialize all actions that involve the
machine state. This simplifies both reasoning about and implementing
the protocol.
%
However, steps that could otherwise be taken in parallel now need to be
sequentialized, which might hurt performance. For the cases where this
sequentialization would severely affect protocol performance,
we employ a (to our knowledge) novel
technique to parallelize the progression of the state machine on the
mainchain.
% MF: was:
%
%However, it comes at the cost that steps that could be
%taken in parallel now need to be sequentialized, which might hurt
%performance. In cases where this sequentialization severely affects
%the performance of the protocol we employ a---to our knowledge---novel
%technique to parallelize the progression of the state machine on the
%mainchain.

We use this technique % in the simplified protocol
to parallelize the construction of the initial UTxO set of the head.
Without parallelization, all $n$ head members would have
to post their commit transactions (their portion of the
initial UTxO set) in sequence, requiring a linear chain of $n$
transactions, %to incrementally build the initial head UTxO set
each for one state transition at a time.
Instead, we make the state machine consume all $n$ commit transactions
in a single state transition. In Fig.~\ref{fig:SM_states_basic}, we
represent this in the following way:
the transaction representing state $\stInitial$ connects
to the transaction representing state $\stOpen$ not just via the
$\mtxCCom$ state transition, but also via a set of commit transactions
(one for each head member).

This requires some extra care. We want to ensure that each
head member posts exactly one commit transaction and that the $\stOpen$
transaction faithfully collects all commit
transactions. We gain this assurance by issuing a single non-fungible
token to each head member---we call this the \emph{participation
  token}.  This token must flow through the commit transaction of the
respective head member and the $\stOpen$ transaction, to be valid, must
collect the full set of participation tokens. We may regard the
participation token as representing a \emph{capability} and
\emph{obligation} to participate in the head protocol.


\subsection{The head protocol}
\label{sec:overview_hp}

The head protocol starts with an initial set $\Uinit$ of UTxOs that
is identical to the UTxOs locked onchain.

\dparagraph{Transactions and local UTxO state.}
The protocol \emph{confirms} individual transactions in full
concurrency by collecting and distributing multisignatures on each
issued transaction separately.
As soon as such a transaction is confirmed, it irreversibly becomes
part of the head UTxO state evolution---the transaction's outputs
are immediately spendable in the head, or can be safely transferred
back onchain in case of a head closure.

Each party maintains their view of the local UTxO state $\barmL$, which
represents the current set of UTxOs evolved from the initial UTxO set
$\Uinit$ by applying all transactions that have been confirmed so far
in the head. As the protocol is asynchronous the parties' views
of the local UTxO state generally differ.

\dparagraph{Snapshots.}
The above transaction handling would be enough to evolve the head state.
However, an eventual onchain decommit would have to transfer the
full transaction history onchain as there would be no other way
to evidence the correctness of the UTxO set to be restored onchain.

To minimize local storage requirements and allow for an
onchain decommit that is independent of the transaction
history, UTxO snapshots $\Uset_1,\Uset_2,\ldots$ are continuously
generated. For this, a \emph{snapshot leader} requests his
view of the confirmed state $\barmL$ to be multisigned as a new
snapshot---the first head snapshot corresponding to the initial
state $\Uinit$. A snapshot is considered \emph{confirmed} if it
is associated with a valid multisignature.

In contrast to transactions, the snapshots are generated
sequentially.
%the new snapshot
%is the last snapshot whereupon all transactions are
%applied that the snapshot leader sees as confirmed but not yet
%processed by the last snapshot.
To have the new snapshot $\Uset_{i+1}=\barmL$ multisigned, the leader does
not need to send his local state $\Uset_{i+1}$, but only indicate,
by hashes, the set of (confirmed) transactions to be applied to $\Uset_i$
in order to obtain $\Uset_{i+1}$.

The other participants sign the snapshot as soon as they have (also) seen
the transactions confirmed that are to be processed on top of its predecessor
snapshot: a party's confirmed state is always ahead of the latest
confirmed snapshot.

As soon as a snapshot is seen confirmed, a participant can safely
delete all transactions that have already been processed into it as the
snapshot's multisignature is now evidence that this state once existed
during the head evolution.

\dparagraph{Closing the head.}
A party that wants to close the head decommits his confirmed state
$\barmL$ by posting, onchain, the latest seen confirmed snapshot
$\Uset_\ell$ together with those confirmed transactions that have not
yet been processed by this snapshot.  During the
subsequent contestation period, other head members can post their own
local confirmed states onchain.
