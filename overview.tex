\section{Protocol Overview}\label{sec:overview}

\todo{maybe change terminology (mainchain -> L1, offchain -> L2)}

The Hydra Head protocol provides functionality to lock a set of UTxOs on a
blockchain, referred to as the \emph{mainchain}, and evolve it inside a
so-called offchain \emph{head}, independently of the mainchain. At any point,
the head can be closed with the effect that the locked set of UTxOs on the
mainchain is replaced by the latest set of UTxOs inside the head. The protocol
guarantees full wealth preservation: no generation of funds can happen offchain
(inside a head) and no responsive honest party involved in a head can ever lose
any funds other than by consenting to give them away.

The advantage of head evolution from a liveness viewpoint is that, under good
conditions, it can essentially proceed at network speed, thereby reducing
latency and increasing throughput in an optimal way. At the same time, the head
protocol provides the same smart-contract capabilities as the mainchain.

\subsection{The big picture}\label{sec:overview_bp}

To create a head-protocol instance, any party may take the role of an
\emph{initiator} and ask a set parties, the \emph{head members}, to
participate in the head by exchanging public keys and agreeing on other protocol parameters. 
This public-key material is used both
for the authentication of head-related onchain transactions that
are restricted to head members (e.g., a non-member is not allowed
to close the head) and for multisignature-based event confirmation
in the head.

The initiator then establishes the head by submitting an
\emph{initial} transaction to the mainchain that contains the
head parameters and forges special \emph{participation tokens}
identifying the head members by assigning each token to the public key
distributed by the respective party during the the setup phase.
The initial transaction also initializes a state machine
(see Fig.~\ref{fig:SM_states_basic}) for the head instance that manages
the ``transfer'' of UTxOs between mainchain and head and forges one State Thread ($ST$) token marking the head output.

Once the initial transaction appears on the mainchain, establishing
the initial state $\stInitial$, each head member can attach a
\mtxCom{} transaction, which locks (on the mainchain) the UTxOs that
the party wants to commit to the head.

The commit transactions are subsequently collected by the \mtxCCom{} transaction
causing a transition from $\stInitial$ to $\stOpen$. Once the $\stOpen$ state is
confirmed, the head members start running the layer two \emph{head protocol},
which evolves the initial UTxO set (the union over all UTxOs committed by all
head members) independently of the mainchain. For the case where some head
members fail to post a \mtxCom{} transaction, the head can be aborted by going
directly from $\stInitial$ to $\stFinal$.

The head protocol is designed to allow any head member
at any point in time to produce, without interaction, a certificate 
for the current head UTxO set.
This certificate is created from the latest confirmed snapshot,
specifically from its snapshot number and the respective multisignature.
Using this certificate, the head member
may advance the state machine to the $\stClosed$ state.

Once in $\stClosed$, the state machine grants parties a
\emph{contestation period}, during which each party may (one single
time) contest the closure by providing the certificate for a newer head
UTxO set.  Contesting leads back to the state $\stClosed$.
After the contestation period has elapsed, the state machine may proceed
to the $\stFinal$ state.  The state machine enforces that the
outputs of the transaction leading to $\stFinal$ correspond exactly to
the latest UTxO set seen during the contestation period.


\subsection{The \todo{layer one?}mainchain state machine}\label{sec:overview_mc}

The mainchain part of the Hydra protocol fulfills two principal
functions: (1) it locks the mainchain UTxOs committed % by head members
to the head while the head is active and (2) it facilitates the
settlement of the final head state back to the mainchain after the
head is closed. In combination, these two functions effectively result
in replacing the initial head UTxO set by the final head UTxO set on
the mainchain in a manner that respects but does not persist the
complete set of head transactions.

\begin{figure}[t!]

  \centering

  \includegraphics[width=\columnwidth/3*2]%
  {fig/SM_states_basic_cropped.pdf}
  \caption{Mainchain state diagram for this version of the Hydra protocol.}
  \label{fig:SM_states_basic}

\end{figure}



%%% Local Variables:
%%% mode: latex
%%% TeX-master: "main"
%%% End:


The state machine (Fig.~\ref{fig:SM_states_basic}) implementing
the mainchain protocol comprises the four states $\stInitial$,
$\stOpen$, $\stClosed$, and $\stFinal$, where the first two realize
the first function (locking the initial UTxO set) and the
second two realize the second function (settling the final UTxO set on
the mainchain).

Note that committing UTxOs to the head is not done in this state machine, to avoid overly sequentializing these steps. Instead, $\mtxCom$ transactions can be made in parallel and the state machine later
consumes all $n$ commit transactions in a single state transition. In
Fig.~\ref{fig:SM_states_basic}, we represent this in the following way: the
transaction representing state $\stInitial$ connects to the transaction
representing state $\stOpen$ not just via the $\mtxCCom$ state transition, but
also via a set of commit transactions (one for each head member).

This requires some extra care. We want to ensure that each
head member posts exactly one commit transaction and that the $\stOpen$
transaction faithfully collects all commit
transactions. We gain this assurance by issuing a single non-fungible
token to each head member---we call this the \emph{participation
  token}.  This token must flow through the commit transaction of the
respective head member and the $\stOpen$ transaction, to be valid, must
collect the full set of participation tokens. We may regard the
participation token as representing a \emph{capability} and
\emph{obligation} to participate in the head protocol.


\subsection{The Coordinated Head protocol}\label{sec:overview_hp}

The actual Head protocol starts after the initialization phase with an initial
set of UTxOs that is identical to the UTxOs locked onchain via the \emph{commit}
and \emph{collectCom} transactions.

The protocol distributes and \emph{collects} individual transactions in full concurrency,
while each party maintains their view of the local UTxO state. That is, the current set of UTxOs evolved from the initial UTxO set
by applying transactions as they are received from the other parties.
As the protocol is asynchronous the parties' views of the local UTxO state generally differ.

To confirm transactions and allow for an onchain decommit of the resulting UTxO set
without needing the whole transaction history, UTxO snapshots $\Uset_1,\Uset_2,\ldots$ are
strictly sequentially created. The first snapshot corresponds to the initial UTxO set.

For this, a \emph{snapshot leader} requests his view of
a new confirmed state to be multisigned as a new snapshot. The leader does not need to send his local state,
but only indicate, by hashes, the set of transactions to be included in order to obtain the to-be-snapshotted UTxO set.

The other participants sign the snapshot as soon as they have (also) seen the
transactions that are to be processed on top of its preceding snapshot: a
party's local state is always ahead of the latest confirmed snapshot.

Signatures are broadcast and aggregated by each party. When all signatures parts of the
multi-signatures are received and verified a snapshot is considered confirmed.
As a consequence, a participant can safely delete all transactions that have been processed into it as the
snapshot's multisignature is now evidence that this state once existed during the head evolution.

A party can close the head using their last known confirmed snapshot using the $\mtxClose$ transaction.
During the subsequent contestation period, other head members can post their own view of the confirmed states onchain.
Eventually, the latest snapshot with the highest snapshot number will become the final UTxO set that is distributed on the mainchain.

\subsection{Differences}

In the Coordinated Head protocol, off-chain consensus is simplified by not having transactions
confirmed concurrently to the snapshots (and to each other) but having the snapshot leader propose,
in their snapshot, a set of transactions for explicit confirmation. The parties' views of confirmed
transactions thus progress in sync with each other (once per confirmed snapshot), thus simplifying
the close/contest procedure on the mainchain. Also, there is no need for conflict resolution as
in Appendix~B of~\cite{hydrahead20}. In summary, the differences to the original Head protocol in~\cite{hydrahead20} are:

\begin{itemize}
  \item No hanging transactions due to `coordination'.
  \item No acknowledgement nor confirmation of transactions.
  \item No confirmation of snapshots (two-round confirmation by local acknowledgement).
\end{itemize}
\todo{explain why?}

%%% Local Variables:
%%% mode: latex
%%% TeX-master: "main"
%%% End:
